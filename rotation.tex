\documentclass[a4paper,10pt]{amsart}
\usepackage{kubik}

\begin{document}
\title{On the total curvature of \\
minimizing geodesics on convex surfaces}
\author{Nina Lebedeva}
\address{N. Lebedeva\newline\vskip-4mm
Math. Dept.
St. Petersburg State University,
Universitetsky pr., 28, 
Stary Peterhof, 
198504, Russia.
\newline\vskip-4mm
Steklov Institute,
27 Fontanka, St. Petersburg, 
191023, Russia.}
\email{lebed@pdmi.ras.ru}
\author{Anton Petrunin}
\address{A. Petrunin\newline\vskip-4mm
Math. Dept. PSU,
University Park, PA 16802,
USA}
\email{petrunin@math.psu.edu}
\thanks{N.~Lebedeva was partially supported by RFBR grant 
14-01-00062.}
\thanks{A.~Petrunin was partially supported by NSF grant DMS 1309340.}


\date{}

\begin{abstract}
We give a universal upper bound for the variation of turn of minimizing geodesic on a convex surface in the Euclidean space.
\end{abstract}
\maketitle


\section{Introduction}

Denote by $\EE^3$ the 3-dimensional Euclidean space.

Recall that the \emph{turn} of a curve $\gamma\:\II\to \EE^3$ 
(briefly $\turn\gamma$)
is defined as supremum of sum of exterior angles 
for the broken lines inscribed in $\gamma$.
If $\gamma$ is smooth and equipped with the natural parameter, 
then 
\[\turn\gamma=\int\limits_\II \kappa(t)\cdot dt,\]
where $\kappa(t)=|\gamma''(t)|$ is the curvature of $\gamma$ at $t$.
By that reason the turn of curve is also called its \emph{total curvature}.

\begin{thm}{Main theorem}\label{thm:main}
If $K$ is a closed convex set in the 3-dimensional Euclidean space,
$\Sigma$ is the surface of $K$ 
and $\gamma$ be a minimizing geodesic in $\Sigma$
then 
\[\turn\gamma\le \omega,\]
where $\omega$ is a universal real constant.
\end{thm}

The question was stated in \cite{AH-PSV}, \cite{pach} and \cite{BKZ},
but we have learned it from Dmitry Burago only few years ago.

Let us briefly discuss the related results.

\begin{itemize}
\item In \cite{liberman}, Liberman gives a bound on the turn of short geodesic in terms of the ratio diameter and inradius of $K$.
In the proof he use now so called Liberman's lemma \ref{lem:liberman} discussed below.
This statement was rediscovered in \cite{BKZ}.
\item In \cite{usov}, Usov gives the optimal bound for turn of geodesic on the graph of $\ell$-Lipscitz convex function. 
Namely, he proves that if $f\:\RR^2\to\RR$ is $\ell$-Lipschitz and convex then any 
geodesic in its graph 
\[\Gamma_f=\set{(x,y,z)\in \RR^3}{z=f(x,y)}\] 
has turn at most $2\cdot \ell$.
This statement was also rediscovered in \cite{BKZ}.
Yet an amusing generalization of Usov's result is given by Berg in \cite{berg}.
\item In \cite{pogorelov}, Pogorelov conjectured that any the spherical image of geodesic on convex surface has to be contructable.
It is easy to see that the length of spherical image of geodesic can not be smaller than its turn, so this conjecture (if true) would be stronger than Liberman's theorem.
Counterexamples were found indepenently by Milka in \cite{milka}, 
Usov in \cite{usov-conj-pog} 
and much later by Pach in \cite{pach}.
\item In \cite{BKZ},
B{\'a}r{\'a}ny,
Kuperberg, 
and Zamfirescu 
have constructed a corkscrew minimizing geodesic on a closed hypersurface;
that is a minimizing geodesic which twists around given line arbitrary many times.

\item In the same paper they also constructed a minimizing geodesic on a convex surface in $\RR^3$
with turn bigger that $2\cdot\pi$.
(Note that $2\cdot\pi$ is the optimal bound for the analogous problem in the plane.)
\end{itemize}

\parbf{Plan of the proof.}
We prove is divided in three steps.

First we prove a sequence of propositions which alow us to consider only special case of surfaces and curves.
Namely we show that we can assume that
\begin{enumerate}[(i)]
\item\label{smooth}{\it (Proposition \ref{prop:smooth}).} 
The surface $\Sigma$ is $C^\infty$-smooth.
\item{\it (Proposition \ref{prop:almost-const}).}  The $z$-component of $\gamma'$ is positive,
\item\label{graph}{\it (Proposition \ref{prop:graph}).} The surface $\Sigma$ is formed by a graph $z=f(x,y)$ of a smooth convex function $f\:\RR^2\to \RR$.
\end{enumerate}

On the second step we give number of geometric inequalities which relate the angles between $\gamma'(t)$ and with the coordinate axis.

The last step in the proof is purely algebraic, 
here we combine the obtained inequalities to give a universal bound on the turn of $\gamma_n$ whcih satisfy the conditions  estimate the turn of $\gamma_n$ which satisfy the conditions (\ref{smooth})--(\ref{graph}).


\section{Preliminaries}

Let $\Sigma$ be a convex hypersurface in the Euclidean space.

Given a point $p\in \Sigma$, we will denote by $n_p$ the outer normal vector of $\Sigma$ at $p$;
the map $\Sigma\to\SS^2$ defined as $p\mapsto n_p$ sometimes is called \emph{Gauss map}.

Fix a points $z\notin\Sigma$.
Given a point $p\in \Sigma$,
we say that $p$ lies on light (dark) side from $z$ if 
if $\langle z-p,n_p\rangle\le 0$ (correspondingly $\langle z-p,n_p\rangle\ge 0$).
If $\langle z-p,n_p\rangle= 0$ we say that $p$ lies on the horizon
from $p$.
Note that if $z$ lies inside of $\Sigma$ then all points on $\Sigma$ lie on the dark side from $z$.

Let $\gamma$ be a space curve 
parametrized by length.
Fix a point $z\notin\gamma$. 
Let us define \emph{Liberman's development} of $\gamma$ with respect to $z$ as the unit-speed plane cure $\tilde\gamma_z$ such that the direction $\tilde\gamma_z(t)$
changes counterclockwise as $t$ changes
and
$|\tilde\gamma_p(t)|=|\gamma(t)-z|$ for any $t$.

The Liberman's development $\tilde\gamma_z$ is called convex concave at $\tilde\gamma_z(t)$ if there the curvelinear triangle ??? 


\begin{thm}{Liberman lemma}\label{lem:liberman}
Let $\Sigma$ be a convex surface in the Euclidean space 
$z\not\in\Sigma$ and $\gamma$ be a unit-speed geodesic in $\Sigma$.
Then the development $\tilde\gamma_z$ is locally convex (concave) 
at the points on dark (light) side of $\Sigma$ with respect to $z$.
\end{thm}

Assume $\gamma\:[0,\ell]\to \Sigma$ is a unit-speed curve in the space.

The vector $\gamma''(t)$ is the curvature vector of $\gamma$ at $t$.
The turn of $\gamma$ can be defined as 
\[\turn\gamma\df\int\limits_0^\ell|\gamma''(t)|\cdot dt.\]

The turn of $\tilde\gamma_z$ is called the turn of $\gamma$ in the direction of $z$ and denoted as $\turn_z\gamma$
Given a point $z$, let us define the turn of $\gamma$ in the direction of $z$ as
\[\turn_z\gamma
\df
\int\limits_0^\ell
\left|
\langle\gamma''(t),\tfrac{z-\gamma(t)}{|z-\gamma(t)|}\rangle
\right|
\cdot dt.\]

\begin{thm}{Key Lemma}
\label{lem:key}
Let $\gamma\:[0,\ell]\to \Sigma$ be a geodesic on the convex surface in the Euclidean space 
and $u\in\SS^2$.
Assume that $0=t_0<t_1<\dots<t_n=\ell$ be the values such that each arcs $\gamma|_{[t_{i-1},t_i]}$ alternating light and dark side of $\Sigma$ with respect to $u$.
Set $\alpha_i=\measuredangle(\dot\gamma(t_i),u)$
Then 
\[\turn_u\gamma=|\sum_i(-1)^i\alpha_i|.\]

Moreover, if $1<i<n$ 
and $\Omega_i$ denotes the domain of $\Sigma$ bounded by the arc $\gamma|_{[t_{i-1},t_i]}$ and the $u$-horizon then 
\[|\alpha_i-\alpha_{i-1}|\le \curv\Omega_i,\]
where $\curv\Omega_i$ denotes the total curvature of $\Omega_i$.
In particular,
\[\turn_u\gamma\le 4\cdot\pi+\sum_i\curv\Omega_i.\]
\end{thm}

\parbf{Remarks.}
Clearly $\turn_z\gamma\le \turn \gamma$ for any curve $\gamma$ in $\Sigma$.

On the other hand given few points $z_i$ which do not lie in one plane
one can estimate $\turn\gamma$ 
in terms of $\turn_{z_i}\gamma$ 
the distances between $z_i$ and the maximal distance to $\gamma$.

Let $N=N(\Sigma,\gamma, u)$ be the maximal integer such that at most $N$ of the domains $\Omega_i$ intersect at one point.
Note that from \cite{BKZ}, it follows that the value $N$ can take arbitrary large value.
The number $N$ can be estimated through the maximal rotation number of subarcs of $\gamma$ with respect to the lines.
In particular the turn of geodesic $\gamma$ can be bounded in terms of maximal rotation number of subarcs of $\gamma$ around the lines.
The later was claimed in \cite{BKZ} without a proof.

Then
\[\sum_{i=2}^{n-1}\curv\Omega_i
\le 
N\cdot\curv\Sigma
\le 
4\cdot N\cdot\pi.\]
Therefore, we get an estimate
\[\turn_u\gamma
\le 
4\cdot N\cdot\pi+|\alpha_0-\alpha_1|+|\alpha_{n-1}-\alpha_n|
\le
(4\cdot N+2)\cdot\pi.\]
Since the same holds for any vector $u$,
we can taking avarage we get
\[\turn\gamma\le 3\cdot(4\cdot N+2)\cdot\pi.\]


\section{Reduction to the smooth case}

\begin{thm}{Proposition}\label{prop:smooth}
Let $\Sigma$ be a convex surface and $\gamma$ be a geodesic on $\Sigma$.
Then there is a sequence of smooth convex surfaces $\Sigma_n$ converging to $\Sigma$
and an sequence of geodesics $\gamma_n$ in $\Sigma_n$ converging to $\gamma$.
\end{thm}

\parit{Proof.}
Assume $\gamma$ is a minimizing geodesic on a convex surface $\Sigma$.
Assume $\gamma$ parametrized by its length $[0,\ell]$.
We can choose a subinterval $[a,b]\subset [0,\ell]$
such that $0<a$ and $b<\ell$
and $\turn(\gamma|_{[a,b]})$ is arbitrary close to the $\turn \gamma$.

Set $p=\gamma(a)$ and $q=\gamma(b)$.

Assume $\Sigma_n$ be a sequence of smooth convex surfaces converging to $\Sigma$.
and $p_n,q_n\in\Sigma_n$ be a two sequences of points which converge to $p$ and $q$ 
correspondingly.

Denote by $\gamma_n$ a minimizing geodesic from $p_n$ to $q_n$ in $\Sigma_n$.
Note that $\gamma_n$ converges to $\gamma|_{[a,b]}$
as $n\to\infty$.
\qeds


\section{Reduction to a monotonic case}

Arguing by contradiction 
let us assume that 
there is a convex surface with a minimizing geodesic which has arbitrary large turn.

In this we will show that one can assume in addition something about the surface and the curve.
Namely we will prove the following claim.

\begin{thm}{Proposition}\label{prop:almost-const}
Assume Main Theorem does not hold;
that is, there is a sequence of convex surfaces $\Sigma_n$
and a sequence of minimizing geodesic $\gamma_n$ in $\Sigma_n$ such that 
\[\turn \gamma_n\to\infty\ \ \text{as}\ \ n\to\infty.\]

Then we can assume in addition that $\Sigma_n$ is smooth and $\gamma_n'(t)$ is nearly constant.
\end{thm}

\parit{Proof.}
Applying Proposition~\ref{prop:smooth}, we can assume that $\Sigma_n$ are smooth.


Take a sequence of surfaces $\Sigma_n$ with a sequence of minimizing geodesics $\gamma_n$ such that 
\[\turn\gamma_n\to \infty.\]
Denote by $K_n$ the convex body bounded by $\Sigma_n$.

Applying rescaling, we can assume that $\diam\gamma_n=1$ for all $n$.
Passing to a subsequence if necessary, we can also assum that $K_n$ converges in Hausdorff sense, say to $K_\infty$.

By Liberman's theorem we have that $K_\infty$ lies in the plane.

It follows that $\gamma_n$ converges to a line segment in $K_\infty$
or broken line made from two line segments, one on one side and the other on the other side of $K_\infty$.
In the later case we can pass to an arc of $\gamma_n$ so that its turn is still converges to $\infty$ as $n\to\infty$ and the limit curve is a line segment.

We can pass to an arc of $\gamma_n$ and rescale to ensure that 
\[1\le|p_n-\gamma(t)|\le 2\] 
for any $t$ and some fixed point $p_n\in K_n$
while still $\turn\gamma_n\to\infty$ as $n\to \infty$.
In particular 
\[\limsup_{n\to\infty}(\turn_{p_n}\gamma_n)\le \pi.\]

Therefore, passing to a subarc of $\gamma_n$ we can assume that 
$\turn_{p_n}\gamma_n\to0$ and $\turn\gamma_n\to\infty$ as $n\to\infty$.

Passing to a subsequence we can assume that for all $t$, 
the angle between $\gamma'_n(t)$ and $p_n-\gamma_n(t)$ 
converges to a fixed value, 
say $\phi$ 
as $n\to \infty$.

Let us show that we can assume that $\phi=0$.
If $\phi=\pi$ then it is sufficient to change reparametrize each $\gamma_n$ in the opposite direction.
Otherwise repeat the construction for a point $q_n$ instead of $p_n$, 
so that the triangles $\triangle p_nq_n\gamma_n(t)$ are far from being degenerate.
We get that $\gamma_n$ runs in nearly one direction.
Then one can choose $p_n$ near the end of $\gamma_n$.

...
\qeds


\section{Angle estimates}

Fix a $(x,y,z)$-coordinates on the Euclidean space.
The lines parallel to the $z$-axis will be called \emph{vertical};
the lines and planes parallel to $(x,y)$-plane will be called \emph{horisontal}.

Given a vector $\bm{\upsilon}$, we will denote by $\upsilon_x$, $\upsilon_y$ and $\upsilon_z$ its components.
Set 
\[\bm{e}_x=(1,0,0),\ \ \bm{e}_y=(0,1,0)\ \ \text{and}\ \ \bm{e}_z=(0,0,1).\]

Let $\Sigma$ be a convex surface which bounds convex set $K$
and $\gamma\:\II\to \Sigma$ is a minimizing geodesic.

Given $t\in \II$, 
consider the oriented orthonormal frame $\lambda(t),\mu(t),\nu(t)$ 
such that $\nu(t)$ is the outer normal to $\Sigma$ at $\gamma(t)$,
the vector $\mu(t)$ is horizontal and therefore the vector $\lambda(t)$ lies in the plane spanned by $\nu(t)$ and the $z$-axis.

According to Proposition~\ref{prop:almost-const},
 we assume that $\gamma'_z(t)>0$ for any $t$.
In particular, $\nu(t)$ is not vertical and therefore
the frame $(\lambda,\mu,\nu)$ is uniquely defined.

After rotating $(xy)$-plane if necessary, 
we can assume that the border of shadow in the directions of $x$- and $y$-axises 
are smooth curves and $\gamma$ intersects them transversely.

Denote by $t_1,t_2,\dots, t_k$ the time moments in the increasing order 
at which $\gamma$ intersects 
the border of shadow in the direction of $x$-axis.
Note that $\mu(t_n)=\pm e_x$;
define $s_n$ to be $\pm1$ so that
\[\mu(t_n)=s_n\cdot e_x.\]

Further denote by $\alpha(t)$ the signed angle between $\dot\gamma(t)$ and $\lambda(t)$ in the tangent plane at $\gamma(t)$;
we assume that $\alpha(t)$ takes values in $(-\tfrac\pi2,\tfrac\pi2)$.
Set 
\[\alpha_n=\alpha(t_n).\]

Note that  
\begin{align*}
\turn_{\bm{e}_x} \gamma
&\le 
2\cdot\pi+\turn_{\bm{e}_x}\left(\gamma|_{[t_1,t_k]}\right)=
\\
&=2\cdot\pi+\left|\sum s_n\cdot \alpha_n\right|.
\end{align*}

Let us define two more angles.
\begin{itemize}
\item Let $\phi(t)$ be the angle between $\dot\gamma(t)$ and $z$-axis. Set $\phi_n=\phi(t_n)$.
\item Let $\psi(t)$ be the signed angle between $\nu(t)$ and $(x,y)$-plane. 
Set $\psi_n=\psi(t_n)$.
\end{itemize}
Note that 
\[\phi(t)\ge |\psi(t)|\ \ \text{and}\ \  \phi(t)\ge |\alpha(t)|\] 
for any $t$.
In particular 
\[\phi_n\ge |\psi_n|\ \ \text{and}\ \  \phi_n\ge |\alpha_n|\]
for any $n$.

Denote by $\nu^h(t)$ its horizontal projection of $\nu(t)$.
Since $\gamma'_z(t)>0$, 
we have $\nu^h(t)\ne0$ for any $t$.
Define the rotation $\rho_{[a,b]}$  
of the interval $[a,b]\subset\II$
as the
algebraic rotation of $\nu^h(t)$ around the origin in $(xy)$-plane;
say it can be defined by the formula 
\[\rho_{[a,b]}
=
\int\limits_a^b \<\tau(t),\mathrm{J}(\tau'(t))\>\cdot dt,\]
where $\mathrm{J}\:\RR^2\to\RR^2$ denotes the rotation by angle $\tfrac\pi 2$ around the origin and \[\tau\df\tfrac{\nu^h}{|\nu^h|}.\]

\begin{thm}{Claim}\label{clm:rotations}
Let $n=\lceil \sup\set{\rho[a,b]}{[a,b]\subset\II} \rceil$
then \[\turn_{\bm{e}_x}\gamma\le 4\cdot n\cdot\pi.\]

\end{thm}


\begin{thm}{Claim}
Assume that $[b,c]\subset [a,d]\subset \II$ and
$\gamma_z(b)
\ge
\tfrac12\cdot(\gamma_z(a)+\gamma_z(d))$,
$\rho[b,c]\ge 3\cdot \pi$.
Then $\psi(t)>0$ for any $t\ge c$.

In other words the arc $\gamma|_{[c,d]}$ lies on a graph $z=f(x,y)$ of a concave function $f\:\RR^2\to\RR$.
\end{thm}

Note that the last two claims imply the following.

\begin{thm}{Proposition}\label{prop:graph}
Assume Main Theorem does not hold;
that is, there is a sequence of convex surfaces $\Sigma_n$
and a sequence of minimizing geodesic $\gamma_n$ in $\Sigma_n$ such that 
\[\turn \gamma_n\to\infty\ \ \text{as}\ \ n\to\infty.\]

Then we can assume in addition that $\Sigma_n$ is a graph $z=f_n(x,y)$ of a smooth convex function $f_n\:\RR^2\to\RR$
and $\gamma'_z(t)>0$ for any $t\in\II$. 
\end{thm}

In particular, from now on $\psi(t)>0$ for any $t\in\II$.
Note also that by Liberman's lemma $\phi(t)$ is a nondecreasing function on $\II$.

\begin{thm}{Claim}
Let $[a,b]\subset\II$ and $\rho[a,b]\ge 3\cdot\pi$.
Then $\psi(t)\ge \phi(a)$ for any $t\ge b$.
\end{thm}


\begin{thm}{Claim}
Assume  
$\psi(t)\ge \eps>0$ for any $t\in [a,b]$.
Then 
$\alpha(b)-\alpha(a)>\eps\cdot\rho[a,b]$.
In particular, either $|\alpha(a)|>\tfrac12\cdot\eps\cdot\rho[a,b]$ or $|\alpha(b)|>\tfrac12\cdot\eps\cdot\rho[a,b]$.
\end{thm}

Note that above two claims imply the following.

\begin{thm}{Proposition}
Assume $\gamma$ as in Proposition \ref{prop:graph}
and for some $i<j$ we have
\[\left|\sum_{n=i}^js_i\right|=5\]
Then $\phi_j>2\cdot\phi_i$.
\end{thm}



\section{The sum estimate}

There is no geometry in this section.
Here we give an estimate for a sum 
of finite sequence of real numbers 
of a very specific form.


Assume a finite  sign-sequence $\bm{s}=(s_1,\dots, s_k)$
is given;
that is $s_i=\pm1$ for $i$.

We say that a pair of indexes $i< j$
forms an \emph{$\bm{s}$-pair} 
if 
\begin{align*}
\sum_{n=i}^js_n&=0&&
\text{and}&
\sum_{n=i}^{j'}s_n&>0
\end{align*} 
if $i<j'<j$.

Note that for any index $i$ appears in at most one $\bm{s}$-pair.
If you exchange ``$+1$'' and ``$-1$'' in $\bm{s}$ by ``$($'' and ``$)$'' correspondingly then $(i,j)$ is an $\bm{s}$-pair
if and only if the $i$-th bracket forms a pair with $j$-bracket;
in particular for any $\bm{s}$-pair $(i,j)$ we have
\begin{itemize}
\item $s_i=1$; that is, $i$-th braket has to be openning.
 \item $s_j=-1$; that is, $j$-th braket has to be closing.
\end{itemize}

We say that $q$ is the depth of an $\bm{s}$-pair $(i,j)$
(briefly $q=\depth_{\bm{s}}(i,j)$) 
if $q$ is the maximal number such that theis $q$-long nested sequence of $\bm{s}$-pairs starting with $(i,j)$; 
that is a sequence of $\bm{s}$-pairs
$(i,j)=(i_1,j_1),(i_2,j_2),\dots,(i_q,j_q)$ such that
\[i=i_1<\dots<i_q<j_q<\dots<j_1=j.\]

\begin{thm}{Proposition}
Assume that
\begin{itemize}
\item $s_1,\dots, s_k$ is a sign sequence,
\item $0\le K_1\le K_2\le \dots\le K_k$.
\item a sequence $\alpha_1,\dots,\alpha_k$ is such that for $\bm{s}$-pair $(i,j)$, we have
\[|\alpha_i-\alpha_j|\le \depth_{\bm{s}}(i,j)\cdot(K_j-K_i),\]
\item $0\le \phi_1\le\dots\le\phi_k$ such that $\phi_i\ge |\alpha_i|$ for any $i$ and $\phi_j>2\cdot\phi_i$ for any $j>i$ such that $|\sum_{n=i}^js_n|=5$.
\end{itemize}
Then
\[|\sum s_n\cdot \alpha_n|\le 20\cdot( K_k+ \phi_k).\]

\end{thm}

\parit{Proof.}
Note that for arbitrary the $\bm{s}$-pairs $(i,j)$ and $(i',j')$
we have three possibllities:
\begin{itemize}
\item $[i,j]\subset [i',j']$ and in this case $\depth_{\bm{s}}(i,j)<\depth_{\bm{s}}(i',j')$;
\item $[i,j]\supset [i',j']$ and in this case $\depth_{\bm{s}}(i,j)>\depth_{\bm{s}}(i',j')$;
\item $[i,j]\cap [i',j']=\emptyset$.
\end{itemize}
In partcular, if $\depth_{\bm{s}}(i,j)=\depth_{\bm{s}}(i',j')$ then the intervals $[i,j]$ and $[i',j']$ do not overlap.


Therefore if 
\[S_q=\sum_{\depth_{\bm{s}}(i,j)=q} (\alpha_i-\alpha_j)\] 
is the sum for all $\bm{s}$-pairs with depth $q$ then 
\[|S_q|\le q\cdot K_k.\]

Since $s_i=1$ and $s_j=-1$ for any $\bm{s}$-pair $(i,j)$,
we have
\[\alpha_i-\alpha_j=s_i\cdot\alpha_i+s_j\cdot\alpha_j.\]
Therefor sum  $S_1+S_2+S_3+S_4+S_5$  contains some  terms of 
$\sum_{n} s_n\cdot \alpha_n$.
Denote by $G\subset\{1,\dots,k\}$ the remaining indexes.
Let us divide $G$ into 5 groups, say $G_1,\dots,G_5$ 
so $i$ and $j$ go into the same group if 
\[\sum_{n=i}^j s_n\equiv 0\pmod 5.\]

Note that
\begin{align*}
|\sum_{n\in G_m}s_n\cdot \alpha_n|
&\le \sum_{n\in G_m}\phi_n\le
\\
&\le 2\cdot\phi_k;
\end{align*}
the last inequality follows since
$\phi_j>2\cdot \phi_i$
if $i,j\in G_m$ and $i<j$.

Summarizing
\begin{align*}
|\sum_n s_n\cdot\alpha_n|&\le |S_1|+|S_2|+|S_3|+|S_4|+|S_5|+|\sum_{n\in G} s_n\cdot\alpha_n|\le
\\
&\le 15\cdot K_k+10\cdot \phi_k.
\end{align*}
Hence the result follows.
\qeds


\begin{bibdiv}
\begin{biblist}
\bib{AH-PSV}{article}{
   author={Agarwal, Pankaj K.},
   author={Har-Peled, Sariel},
   author={Sharir, Micha},
   author={Varadarajan, Kasturi R.},
   title={Approximating shortest paths on a convex polytope in three
   dimensions},
   journal={J. ACM},
   volume={44},
   date={1997},
   number={4},
   pages={567--584},
   %issn={0004-5411},
   %review={\MR{1481315 (99c:68241)}},
   %doi={10.1145/263867.263869},
}

 \bib{BKZ}{article}{
   author={B{\'a}r{\'a}ny, Imre},
   author={Kuperberg, Krystyna},
   author={Zamfirescu, Tudor},
   title={Total curvature and spiralling shortest paths},
   note={U.S.-Hungarian Workshops on Discrete Geometry and Convexity
   (Budapest, 1999/Auburn, AL, 2000)},
   journal={Discrete Comput. Geom.},
   volume={30},
   date={2003},
   number={2},
   pages={167--176},
   %issn={0179-5376},
   %review={\MR{2007957 (2004h:52009)}},
   %doi={10.1007/s00454-003-0001-z},
}

\bib{berg}{article}{
   author={Berg, I. D.},
   title={An estimate on the total curvature of a geodesic in Euclidean
   $3$-space-with-boundary},
   journal={Geom. Dedicata},
   volume={13},
   date={1982},
   number={1},
   pages={1--6},
   %issn={0046-5755},
   %review={\MR{679213 (84d:53049)}},
   %doi={10.1007/BF00149423},
}

\bib{liberman}{article}{
   author={Liberman, J.},
   title={Geodesic lines on convex surfaces},
   journal={C. R. (Doklady) Acad. Sci. URSS (N.S.)},
   volume={32},
   date={1941},
   pages={310--313},
  % review={\MR{0010994 (6,100g)}},
}

\bib{milka}{article}{
   author={Milka, A. D.},
   title={A shortest path with nonrectifiable spherical representation. I},
%   language={Russian},
   journal={Ukrain. Geometr. Sb.},
   number={16},
   date={1974},
   pages={35--52, ii},
%   review={\MR{0385761 (52 \#6620)}},
}

\bib{pach}{article}{
   author={Pach, J{\'a}nos},
   title={Folding and turning along geodesics in a convex surface},
   journal={Geombinatorics},
   volume={7},
   date={1997},
   number={2},
   pages={61--65},
%   issn={1065-7371},
%   review={\MR{1487759 (98k:52026)}},
}

\bib{pogorelov}{book}{
   author={Pogorelov, A. V.},
   title={Vneshnyaya geometriya vypuklykh poverkhnostei},
%   language={Russian},
   publisher={Izdat. ``Nauka'', Moscow},
   date={1969},
   pages={759},
%   review={\MR{0244909 (39 \#6222)}},
}

\bib{usov}{article}{
   author={Usov, V. V.},
   title={The length of the spherical image of a geodesic on a convex
   surface},
%   language={Russian},
   journal={Sibirsk. Mat. \v Z.},
   volume={17},
   date={1976},
   number={1},
   pages={233--236. %(inside back cover)
   },
%   issn={0037-4474},
%   review={\MR{0405316 (53 \#9110)}},
}

\bib{usov-conj-pog}{article}{
   author={Usov, V. V.},
   title={The three-dimensional swerve of curves on convex surfaces},
%   language={Russian},
   journal={Sibirsk. Mat. \v Z.},
   volume={17},
   date={1976},
   number={6},
   pages={1427--1430, 1440},
%   issn={0037-4474},
%   review={\MR{0442862 (56 \#1237)}},
}

\end{biblist}
\end{bibdiv}


\end{document}