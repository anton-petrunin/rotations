\documentclass[a4paper,10pt]{amsart}
\usepackage{kubik}


\begin{document}
\title{On the total curvature of \\
minimizing geodesics on convex surfaces}
\author{Nina Lebedeva}
\address{N. Lebedeva\newline\vskip-4mm
Math. Dept.
St. Petersburg State University,
Universitetsky pr., 28, 
Stary Peterhof, 
198504, Russia.
\newline\vskip-4mm
Steklov Institute,
27 Fontanka, St. Petersburg, 
191023, Russia.}
\email{lebed@pdmi.ras.ru}
\author{Anton Petrunin}
\address{A. Petrunin\newline\vskip-4mm
Math. Dept. PSU,
University Park, PA 16802,
USA}
\email{petrunin@math.psu.edu}
\thanks{N.~Lebedeva was partially supported by RFBR grant 
14-01-00062.}
\thanks{A.~Petrunin was partially supported by NSF grant DMS 1309340.}


\date{}

\begin{abstract}
We give a universal upper bound 
for the total curvature 
of minimizing geodesic 
on a convex surface 
in the Euclidean space.
\end{abstract}
\subjclass{53A05, 53C45, 53C22}
\maketitle

\section{Introduction}


In this note we give an affirmative answer to the question asked by Dmitry Burago; 
the same question was also stated in \cite{AH-PSV}, \cite{pach} and \cite{BKZ}.
Namely, we prove the following.

\begin{thm}{Main theorem}\label{thm:main}
The total curvature of a minimizing geodesic
on a convex surface in $\RR^3$ can not exceed $1000^{1000}$.
\end{thm}

The value $2\cdot\pi$ is the optimal bound for the analogous problem in the plane.
The total curvature
of minimizing geodesic on a convex surface in $\RR^3$
can exceed $2\cdot\pi$
and the optimal bound 
is expected to be slightly bigger than $2\cdot\pi$.
The former example was constructed by B{\'a}r{\'a}ny,
Kuperberg, 
and Zamfirescu in \cite{BKZ}. 

Let us list other related results.

\begin{itemize}
\item In \cite{liberman}, Liberman gives a bound on the total curvature of short geodesic in terms of the ratio diameter and inradius of $K$.
In the proof he use an analog of Lemma \ref{lem:liberman} discussed below.

\item In \cite{usov}, 
Usov gives the optimal bound for the total curvature of geodesics on the graphs of $\ell$-Lipschitz convex function. 
Namely, he proves that if $f\:\RR^2\to\RR$ is $\ell$-Lipschitz and convex then any 
geodesic in its graph 
\[\Gamma_f=\set{(x,y,z)\in \RR^3}{z=f(x,y)}\] 
has total curvature at most $2\cdot \ell$.
An amusing generalization of Usov's result is given by Berg in \cite{berg}.
\item In \cite{pogorelov}, Pogorelov conjectured 
that the spherical image of geodesic on convex surface has to be (locally) rectifiable.
It is easy to see that the length of spherical image of geodesic can not be smaller than its total curvature, 
so this conjecture (if it would be true) 
would be stronger than Liberman's theorem.
Counterexamples to the different forms of this conjecture were found 
by Zalgaller in \cite{zalgaller},
Milka in \cite{milka}
and Usov in \cite{usov-conj-pog};
these results were partly rediscovered later 
by Pach in \cite{pach}.
\item In \cite{BKZ},
B{\'a}r{\'a}ny,
Kuperberg, 
and Zamfirescu 
have constructed a corkscrew minimizing geodesic on a closed convex surface;
that is a minimizing geodesic which twists around a given line arbitrary many times.
They also rediscovered the results of Liberman and Usov mentioned above.
\end{itemize}

\parbf{Idea of the proof.}
First we show that it is sufficient to estimate total curvature for the minimizing geodesics with almost constant velocity vector, say $\dot\gamma(t)\approx\bm{i}$.

To estimate total curvature in this case it is sufficient 
to estimate the integral
\[\int\langle\ddot\gamma(t),\bm{j}\rangle\cdot dt\] 
for a vector $\bm{j}\perp\bm{i}$.
To understand the idea of this estimate, 
imagine that 
the surface is lighten in the direction of $\bm{j}$,
so it is divided into the dark and light sides 
by a curve $\omega$.
On the diagram you see different combinatorics in which
$\gamma$ meets $\omega$.

\begin{center}
 \begin{lpic}[t(2 mm),b(2 mm),r(0 mm),l(0 mm)]{pics/spiral-and-snake(1)}
\lbl[b]{93,14;$\omega$}
\end{lpic}
\end{center}

In the first case the total curvature is estimated by integral of Gauss curvature of the regions squeezed between $\gamma$ and $\omega$.
The latter follows from the Tongue Lemma \ref{lem:tongue},
which is the hart of our proof.

The second case might look impossible, 
but the corkscrew geodesic constructed in \cite{BKZ} can meets $\omega$ in this order.
Here we show that the total curvature of the twists grows geometrically from a middle twist to the ends and at the ends 
the integral of the full twists can not be bigger than $2\cdot\pi$.
This is enough to estimate the total curvature of the whole geodesic.

The last case is a mixture of first two and it is done by mixing both techniques.

\section{Preliminaries}

\parbf{Total curvature.}
Recall that the \emph{total curvature} of a curve $\gamma\:[0,\ell]\to \RR^3$ 
(briefly $\tc\gamma$)
is defined as supremum of sum of exterior angles 
for the polygonal lines inscribed in $\gamma$.

Note that for a polygonal line $\sigma$, 
its total curvature coincides with the sum of its exterior angles.

If $\gamma$ is a smooth curve with unit-speed parametrization, 
then 
\[\tc\gamma=\int\limits_0^\ell \kappa(t)\cdot dt,\]
where $\kappa(t)=|\ddot\gamma(t)|$ is the curvature of $\gamma$ at $t$.

\begin{thm}{Proposition}\label{prop:semicontinuity}
Assume $\gamma_n\:\II\to\RR^3$ is a sequence of curves  converging pointwise to a curve $\gamma_\infty\:\II\to\RR^3$.
Then 
\[\liminf_{n\to\infty}\tc\gamma_n\ge \tc\gamma_\infty.\]
\end{thm}


\parit{Proof.}
Fix a polygonal line $\sigma_\infty$ inscribed in $\gamma_\infty$.
Let $\gamma_\infty(t_0),\dots,\gamma_\infty(t_k)$
be its vertices 
as they appear on $\gamma_\infty$.
Consider the polygonal line $\sigma_n$ inscribed in $\gamma_n$ with the vertices 
$\gamma_n(t_0),\dots,\gamma_n(t_k)$.
Note that 
\[\tc\sigma_n\to\tc\sigma_\infty\ \ \text{as}\ \ n\to\infty.\]
By the definition of total curvature, 
\[\tc\sigma_n\le\tc\gamma_n.\]
The statement follows since the broken line $\sigma_\infty$ can be chosen in such a way that 
$\tc\sigma_\infty$ is arbitrary close to $\tc\gamma_\infty$.
\qeds

\parbf{Convergence of sets.}
Given a closed set $\Sigma\subset \RR^3$,
denote by $\dist_\Sigma$ the distance function from $\Sigma$;
that is 
\[\dist_\Sigma(x)=\inf\set{|x-y|}{y\in\Sigma}.\]

We say that a sequence of closed sets $\Sigma_n\subset \RR^3$
converges to the closed set $\Sigma_\infty\subset \RR^3$ 
if for any $x\in\RR^3$,
we have
$\dist_{\Sigma_n}(x)\to\dist_{\Sigma_\infty}(x)$ as $n\to\infty$.


\parbf{Convex surfaces.}
By \emph{convex surface} $\Sigma$ in the Euclidean 3-space $\RR^3$ we understand the boundary of a closed convex set $K$ with nonempty interior.
If $K$ is compact we say that the $\Sigma$ is \emph{closed}.

Assume $\Sigma$ is smooth.
If at every point both of $\Sigma$
the principle curvatures are positive, 
we say that $\Sigma$ is \emph{strongly convex}. 

\begin{thm}{Proposition}\label{prop:convegence}
Assume $\Sigma_n$ be a sequence of convex surfaces which converge to a convex surface $\Sigma_\infty$.
Then for any minimizing geodesic $\gamma_\infty$ in $\Sigma_\infty$ there is a sequence of minimizing geodesics $\gamma_n$ in $\Sigma_n$ which pointwise converge to $\gamma_\infty$ as $n\to \infty$.
\end{thm}

\parit{Proof.}
Assume $\gamma_\infty\:[0,\ell]\to\Sigma_\infty$ is parametrized by its arc-length.

Fix a sub-interval $[a,b]\subset (0,\ell)$.
Set $p_\infty=\gamma_\infty(a)$ and $q_\infty=\gamma_\infty(b)$.
Let $p_n,q_n\zz\in\Sigma_n$ be a two sequences of points which converge to $p_\infty$ and $q_\infty$ 
correspondingly.

Denote by $\gamma_n$ a minimizing geodesic from $p_n$ to $q_n$ in $\Sigma_n$.
Note that $\gamma_n$ converges to $\gamma_\infty|_{[a,b]}$
as $n\to\infty$.

Taking the sub-interval  $[a,b]$ closer and closer to $[0,\ell]$
and applying diagonal procedure, we get the result.
\qeds

\section{Liberman's lemma.}

In this section we give a slight generalization 
of the construction given by Liberman in \cite{liberman};
see also \cite{petrunin}.

\parbf{Development.}
Let $\gamma\:[0,\ell]\to\RR^3$ be a curve parametrized by arc-length
and a point $p$ does not lie on $\gamma$.

Assume  $\tilde\gamma_p\:[0,\ell]\to\RR^2$ is a plane curve parametrized by arc-length
and $\tilde p$ is a point in the plane such that 
\[|\tilde p-\tilde\gamma(t)|=|p-\gamma(t)|\]
for any $t\in[0,\ell]$;
moreover, 
the direction from $\tilde p$ to $\tilde \gamma(t)$ changes monotonically (clockwise or counterclockwise). 
Then $\tilde\gamma_p$ is called \emph{development} of $\gamma$ with respect to $p$.

\begin{center}
\begin{lpic}[t(0 mm),b(9 mm),r(0 mm),l(0 mm)]{pics/convex-concave(1)}
\lbl[tl]{23,16;$\tilde\gamma_p(a)$}
\lbl[tr]{4,16;$\tilde\gamma_p(b)$}
\lbl[tl]{14,1;$\tilde p$}
\lbl[ll]{67,18;$\tilde\gamma_p(a)$}
\lbl[rt]{45,18;$\tilde\gamma_p(b)$}
\lbl[tl]{57,1;$\tilde p$}
\lbl[t]{13,-5;Convex development.}
\lbl[t]{56,-5;Concave development.}
\end{lpic}
\end{center}

We say that the development $\tilde\gamma_p$ is \emph{convex} (\emph{concave}) in the interval $[a,b]$
if the arc $\tilde\gamma_p|_{[a,b]}$
cuts from the solid angle $\angle \tilde p^{\tilde\gamma_p(a)}_{\tilde\gamma_p(b)}$
convex bounded (correspondingly unbounded) domain.

We say that $\tilde\gamma_p$ is locally convex (concave) in the interval $[a,b]$
if any point $x\in [a,b]$ admits a closed neighborhood $[a',b']$ in $[a,b]$
such that $\tilde\gamma_p$ is convex (correspondingly concave) in the interval $[a',b']$.

If we pass to the limit of this construction as $p$ moves to infinity along a half-line in the direction of a unit vector $\bm{u}$ then the limit curve is called development of $\gamma$ in the direction $\bm{u}$ and denoted as $\tilde\gamma_{\bm{u}}$.

We can define the development $\tilde\gamma_{\bm{u}}$ directly:
(1) the development $\tilde\gamma_{\bm{u}}\:[0,\ell]\to \R^2$
is parametrized by arc-length,
(2) for a fixed unit vector $\tilde{\bm{u}}\in\RR^2$,
we have
\[\langle \tilde{\bm{u}},\tilde\gamma_{\bm{u}}(t)\rangle
=
\langle  \bm{u},\gamma(t)\rangle\]
for any $t\in [0,\ell]$
and 
(3) the projection of $\tilde\gamma_{\bm{u}}(t)$ to the line normal to $\tilde{\bm{u}}$
is monotonic in $t$.

We can assume that $\tilde{\bm{u}}$ is the vertical vector in the coordinate plane.
In this case we say that $\tilde\gamma_{\bm{u}}$ is concave (convex) in the interval $[a,b]$ 
if the lune bounded by arc $\tilde\gamma_{\bm{u}}|_{[a,b]}$
and the segment $[\tilde\gamma_{\bm{u}}(a)\tilde\gamma_{\bm{u}}(b)]$
is convex and lies above (correspondingly below) the line segment $[\tilde\gamma_{\bm{u}}(a)\tilde\gamma_{\bm{u}}(b)]$.
 
\parbf{Dark and light sides.}
Let $\Sigma\subset\RR^3$ be a convex surface,
$p\in\Sigma$ and $z\ne p$.

We say that $p$ lies on the dark (light) side of $\Sigma$ with from $z$ 
if non of the points $p+t\cdot(p-z)$ lie inside of $\Sigma$ for $t>0$ (correspondingly for $t<0$).
The intersection of dark and light side is called horizon of $z$;
it is denoted by $\omega_z$.

Note that if $z$ lies inside $\Sigma$ 
then all the points on $\Sigma$ lie on the dark side from $z$ and its horizon $\omega_z$ is empty.

If $\Sigma$ is smooth we can define the outer normal vector $\nu_p$ to $\Sigma$ at $p$.
In this case $p$ lies on dark (light) side of $\Sigma$ from $z$
if and only if $\langle p-z,\nu_p\rangle\ge 0$
(correspondingly for $\langle p-z,\nu_p\rangle\le 0$).
If in addition $\Sigma$ is closed and strongly convex then the horizon is empty for $z$ inside $\Sigma$ 
and it is formed by a closed smooth curve for $z$ outside $\Sigma$.

We could also define light/dark sides and horizon in the limit case,
as $p$ escapes to infinity along a half-line in direction $\bm{u}$.

The latter can be defined directly.
We say that a point $p\in\Sigma$ lies on dark (light) side for the unit vector $\bm{u}$ if non of the points $p+\bm{u}\cdot t$ lie inside of $\Sigma$ for $t>0$, (correspondingly $t<0$).
As before the intersection of light and dark side is called horizon of $\bm{u}$ and it is denoted as $\omega_{\bm{u}}$.

In the smooth case the latter means that $\langle \nu_p,\bm{u}\rangle\ge 0$ (correspondingly $\langle \nu_p,\bm{u}\rangle\le 0$).
If in addition $\Sigma$ is closed strongly convex then $\omega_{\bm{u}}$ is a closed smooth curve.

\begin{thm}{Liberman's Lemma}\label{lem:liberman}
Assume $\gamma$ be a geodesic on the convex surface $\Sigma\subset \RR^3$.
Then for any point $z\notin\Sigma$ the development $\tilde\gamma_z$ is locally convex (locally concave) if $\gamma$ lies on dark (correspondingly light) side of $\Sigma$ from $z$.

Similarly for any unit vector $\bm{u}$,
the development $\tilde\gamma_{\bm{u}}$ is locally convex (locally concave) if $\gamma$ lies on dark (correspondingly light) side of $\Sigma$ for $\bm{u}$.

\end{thm}

Note that for any space curve $\gamma$ and any unit vector $\bm{u}$
we have 
\[\tc\tilde\gamma_{\bm{u}}\le\tc\gamma.\]
On the other hand total curvature of few developments gives an estimate 
for the total curvature of the original curve.
For example, if $\bm{i},\bm{j},\bm{k}$ is the orthonormal basis then
\[\tc\gamma
\le
\tc\tilde\gamma_{\bm{i}}+\tc\tilde\gamma_{\bm{j}}+\tc\tilde\gamma_{\bm{k}}.\]

If $\gamma$ lies completely on the dark (or light) side for direction $\bm{u}$
then by Liberman's lemma we get 
\[\tc\tilde\gamma_{\bm{u}}\le \pi.\]
It follows that if $\gamma$ cross the horizons $\omega_{\bm{i}}$, $\omega_{\bm{j}}$ and $\omega_{\bm{k}}$
at most $N$ times then 
\begin{align*}
\tc\gamma
&\le
\tc\tilde\gamma_{\bm{i}}
+\tc\tilde\gamma_{\bm{j}}
+\tc\tilde\gamma_{\bm{k}}
\le
\\
&\le(N+1)\cdot \pi.
\end{align*}
Therefore, 
to violate Main Theorem 
$\gamma$ has to cross the  horizons $\omega_{\bm{i}}$, $\omega_{\bm{j}}$ and $\omega_{\bm{k}}$ huge number of times.

\section{Curvature of development}\label{sec:curv-develop}

Let $\Sigma\subset\RR^3$
be a closed smooth strongly convex surface
and $\gamma\:[0,\ell]\to \Sigma$ be a unit-speed geodesic.
Assume that for some unit vector $\bm{u}$,
the geodesic $\gamma$ cross the horizon $\omega_{\bm{u}}$ transversely at 
$t_0<\dots <t_k$.
Set $\alpha_i=\measuredangle(\dot\gamma(t_i),\bm{u})-\tfrac\pi2$ for each $i$.
Note that $|\alpha_i|\le\tfrac\pi2$.

The values $t_i$ and  $\alpha_i$ 
will be called correspondingly \emph{meeting moments} 
and \emph{meeting angles}
of the geodesic $\gamma$ with the horizon $\omega_{\bm{u}}$.

Let us introduce new notation
\[\tc_{\bm{u}}\gamma\df\tc\tilde\gamma_{\bm{u}}.\]
From Liberman's lemma \ref{lem:liberman},
we get the following.

\begin{thm}{Corollary}\label{cor:liberman}
Let $\Sigma\subset\RR^3$
be a strongly convex smooth surface,
$\gamma\:[0,\ell]\to \Sigma$ be a unit-speed geodesic
and $\bm{u}$ is a unit vector.
Assume that  $\gamma$ cross the horizon $\omega_{\bm{u}}$ transversely
and 
$t_0<\dots <t_k$ be its meeting moments 
and $\alpha_0,\dots,\alpha_k$ be its meeting angles with the horizon $\omega_{\bm{u}}$.
Then
\[\tc_{\bm{u}}\gamma
\le 3\cdot\pi
+
2\cdot\left|\alpha_0-\alpha_1
+\dots +(-1)^k\cdot\alpha_k\right|
\]

\end{thm}

As you will see further, 
in order to find the needed estimate the total curvature
of geodesic we will get an upper bound for the sum
\[\left|\alpha_0-\alpha_1
+\dots +(-1)^k\cdot\alpha_k\right|.\]
Finding  such an upper bound is the most important ingredient in the proof of the Main Theorem.


\parit{Proof.}
By Liberman's lemma,
\[\tc_{\bm{u}}(\gamma|_{[t_{i-1},t_i]})=\pm(\alpha_{i-1}-\alpha_i)\]
where the sign is ``$+$'' 
if  $\gamma_{[t_i,t_{i+1}]}$ lies on the dark side 
and ``$-$'' if it lies on the light side from $\bm{u}$.
Summing all this up we get
\[\tc_{\bm{u}}(\gamma|_{[t_0,t_{k}]})
=
\left|\alpha_0
-2\cdot\alpha_1
+\dots+(-1)^{k-1}\cdot2\cdot\alpha_{k-1}+(-1)^k\cdot\alpha_k\right|.\]
By Liberman's lemma we also have
\[\tc_{\bm{u}}(\gamma|_{[0,t_0]}),\tc_{\bm{u}}(\gamma|_{[t_k,\ell]})\le \pi \]
Since $\alpha_0,\alpha_k\le\tfrac\pi2$, the statement follows.
\qeds 

If $\Sigma$ is a surface in $\RR^3$ and $p\in\Sigma$
we denote by $K_p$ the Gauss curvature of $\Sigma$ at $p$.

\begin{wrapfigure}{r}{37 mm}
\begin{lpic}[t(-5 mm),b(0 mm),r(0 mm),l(0 mm)]{pics/tongue(1)}
\lbl[t]{34,12;{$\omega_{\bm{u}}$}}
\lbl[lb]{26,24;{$\gamma$}}
\lbl[t]{1,11;{\tiny $\gamma(t_i)$}}
\lbl[t]{15,11;{\tiny $\gamma(t_i)$}}
\end{lpic}
\end{wrapfigure}
 
Assume $a,b$ be the meeting moments of minimizing geodesic $\gamma$ with $\omega_{\bm{u}}$.
The arc $\gamma|_{[a,b]}$ will be called \emph{$\omega_{\bm{u}}$-tongue}
if there is an immersed disc 
$\iota\:\DD\looparrowright\Sigma$ 
such that the closed curve $\iota|_{\partial D}$ is formed by joint of the arc $\gamma|_{[a,b]}$ and an arc of $\omega_{\bm{u}}$.
In this case the immersion $\iota$ is called the \emph{disc of the tongue}.

\begin{thm}{Tongue Lemma}\label{lem:tongue}
Let $\bm{u}$ be a unit vector,
$\gamma\:[a,b]\to\Sigma$ 
be a minimizing geodesic on the strongly convex surface $\Sigma\subset\RR^3$ which is a $\omega_{\bm{u}}$-tongue
and 
$\iota\:\DD\looparrowright\Sigma$
is its disc.

Then 
\[\int\limits_{\DD} K_{\iota(x)}\cdot d_{\iota(x)}\area_\Sigma\] takes one of the value
\[\alpha-\beta,
\  -\alpha+\beta,
\ \pi-\alpha-\beta,
 \pi+\alpha+\beta \pmod{2\cdot\pi}.
\]

In particular 
\[\bigl|\alpha-\beta\bigr|
\le
\int\limits_{\DD} K_{\iota(x)}\cdot d_{\iota(x)}\area_\Sigma.
\eqlbl{eq:key2}\]
If in addition the image $\iota(\DD)$ lies  completely in the dark or light side for $\bm{u}$ then
\[\tc_{}\gamma
\le 
\int\limits_{\iota(\DD)} K_p\cdot d_p\area_\Sigma.\eqlbl{eq:key3}\]
\end{thm}

\parit{Proof.}
Since $\gamma$ is a geodesic, 
the parallel translation along $\gamma$ 
sends $\dot\gamma(a)$ to $\dot\gamma(b)$.

Note also that $\bm{u}$ belongs to the tangent plane to $\Sigma$ at any point on the horizon $\omega_{\bm{u}}$;
in particular the $\bm{u}$ extends to a parallel tangent vector field on $\omega_{\bm{u}}$.

It follows that parallel translation along $\iota|_{\partial\DD}$ 
rotates the tangent plane by angle 
\[\pm(\tfrac\pi2+\alpha)\pm(\tfrac\pi2+\beta)\]
To prove the main statement of the lemma,
it remains to apply Gauss--Bonnet formula.

Denote by $R$ the right hand side in \ref{eq:key2}.
Note that $R\ge 0$ 
and $|\alpha|,|\beta|\le \tfrac\pi2$.
From the main statement of the lemma it follows then that the minimal possible value for $R$ is $\bigl|\alpha-\beta\bigr|$.

To prove \ref{eq:key3}, note that 
in this case $\iota$ is an embedding.
Further note that the spherical image of the dark side of $\Sigma$ is hemisphere.
Therefore $2\cdot\pi$ is the integral of Gauss curvature along the dark side.
It follows that
\[\int\limits_{\DD} K_{\iota(x)}\cdot d_{\iota(x)}\area_\Sigma
=
\int\limits_{\iota(\DD)} K_p\cdot d_p\area_\Sigma
<2\cdot\pi.\]
By Liberman's lemma the statement follows.
\qeds

\section{Almost straight arcs}

Let $\eps>0$.
A curve $\gamma\:[a,b]\to\RR^3$ will be called $\eps$-straight
if 
\[(1-\eps)\cdot\length \gamma\le |\gamma(b)-\gamma(a)|\]

\begin{thm}{Lemma}\label{lem:eps-straight}
Assume $\eps>0$ 
and $n$ is a positive integer such that
$n\cdot\eps>2$.
Then any minimizing geodesic
on a convex surface $\Sigma$ in $\RR^3$
can be subdivided 
into $\eps$-straight arcs $\gamma_1,\dots,\gamma_n$.
\end{thm}

\parit{Proof.}
Let $\theta\in(0,\pi)$
be such that 
\[1-\cos\theta=\eps.\]

Assume two points $p$ and $q$ lie on the convex surface $\Sigma$.
Denote by $\nu_p$ and $\nu_q$ the outer normal vectors at $p$ and $q$ correspondingly.
Note that if 
\[\measuredangle(\nu_p,\nu_q)\le 2\cdot\theta\]
then any minimizing geodesic from $p$ to $q$ on $\Sigma$
is $\eps$-straight.

Let $\gamma\:[0,\ell]\to \Sigma$ be a minimizing geodesic parametrized by its arc-length.

Assume $\gamma_{[t,\ell]}$ is not $\eps$-straight.
Set $t'$ to be the maximal value in $[t,\ell)$ such that the  interval $[t,t']$ is $\eps$-straight.
Consider a sequence $0=t_0<t_1<\dots<t_n<\ell$
such that $t_{i+1}=t_i'$ for each $i$.
Denote by $\nu_i$ the outer unit normal vector to $\Sigma$ at $\gamma(t_i)$. 
From above we get
\[\measuredangle(\nu_i,\nu_j)\ge2\cdot\theta\] for all $i$ and $j$.
In other words, the open balls $\mathrm{B}_\theta(\nu_i)$ do not overlap 
in $\SS^2$.

It remains to note that 
\[\area[\mathrm{B}_\theta(\nu_i)] =2\cdot\pi\cdot\eps
 \ \ \text{and}\ \ 
 \area\SS^2=4\cdot\pi.
\]
Hence the result follows.
\qeds

\begin{thm}{Corollary}
Assume $\gamma\:[0,\ell]\to \Sigma$ is a unit-speed minimizing geodesic on the convex surface $\Sigma$ in $\RR^3$.
Then $\diam \gamma\ge \tfrac\ell{10}$.
\end{thm}

\parit{Proof.}
Apply Lemma~\ref{lem:eps-straight} for $\eps=\tfrac12$.
\qeds


\section{An arc in almost constant direction}

\begin{thm}{Proposition}\label{prop:almost-const}
For any $\eps>0$ there is $\delta>0$ such that the following holds.

If $\gamma\:[a,b]\to\Sigma$ is a minimizing geodesic 
on a smooth strongly convex surface $\Sigma$ in $\RR^3$
then there is an interval $[a',b']\subset[a,b]$
such that 
\[\tc(\gamma|_{[a',b']})>\delta\cdot\tc\gamma.\]
and 
\[\measuredangle(\dot\gamma(t),\bm{u})<\eps\] 
for a fixed unit vector $\bm{u}$
and any $t\in[a',b']$.

Moreover, if $\eps=\tfrac1{10}$ then one can assume $\delta=\tfrac1{100^{100}}$.
\end{thm}

In the proof we will need the following two lemmas.

\begin{thm}{Lemma}\label{lem:almost-const-angles}
For any $\eps$ there is $\delta>0$ such that the following holds.

Assume $\gamma$ is a curve,
$\bm{v}_1$ and $\bm{v}_2$  be two vectors in $\RR^3$
and $0\le\alpha_1,\alpha_2\le\pi$ be such that
\begin{align*}
\eps
&<\measuredangle(\bm{v}_1,\bm{v}_2)<\pi-\eps
\\
\alpha_i-\delta
&<\measuredangle(\bm{v}_i,\dot\gamma(t))< \alpha_i+\delta
\end{align*}
then there is a vector $\bm{u}$ such that
$\measuredangle(\bm{u},\dot\gamma(t))<\eps$.

Moreover if $\eps<\tfrac1{10}$ 
then one can take $\delta=\eps^{10}$.
\end{thm}

The proof of the lemma above 
is straightforward computation;
we omit it.

\begin{thm}{Lemma}\label{lem:almost-const}
For any $\eps>0$ there is $\delta>0$
such that the following holds.

Let $\gamma\:[a,b]\to\Sigma$ 
be an $\delta$-straight minimizing geodesic 
on a smooth strongly convex surface $\Sigma$ in $\RR^3$.
Set $\bm{v}_\gamma=\gamma(b)-\gamma(a)$.
Then there in a sub-interval $[a',b']$ in $[a,b]$
such that 
\[\tc(\gamma|_{[a',b']})\ge\delta\cdot\tc\gamma.\]
and 
\[\alpha-\eps
\le
\measuredangle(\dot\gamma(t),\bm{v}_\gamma)
\le\alpha+\eps\] 
for some fixed $\alpha$
and
any $t\in[a',b']$.

Moreover if $\eps<\tfrac1{10}$ one can take $\delta=\eps^{10}$.
\end{thm}

\parit{Proof.}
Without loss of generality we can assume that
$a=0$, $b=2$ and 
\[\tc(\gamma|_{[1,2]})
\ge
\tfrac12\cdot\tc\gamma.\]

Set $p=\gamma(0)$.
Let $\theta\in(0,\pi)$ be such that $1-\cos\theta=\delta$.
Note that 
$$\measuredangle(\bm{v}_\gamma,\gamma(t)-p)
\le
\measuredangle(\tilde\gamma_p(1)-\tilde p,\tilde\gamma_p(2)-\tilde p)
<
\theta
\eqlbl{eq:angle1}$$
for any $t\ge 1$.

By Liberman's lemma 
\[\tc_p(\gamma|_{[1,2]})< \pi+\theta.\]
Assume $N=\lceil\tfrac\pi\theta+1\rceil$.
Then we can subdivide $\gamma|_{[1,2]}$ into $N$ arcs 
 $\gamma_1,\gamma_2,\dots,\gamma_N$ such that
\[\tc_p(\gamma_n)\le \theta\eqlbl{eq:tcp}\]
for each $n$.

From \ref{eq:angle1} and \ref{eq:tcp},
it follows that for each $n$, there is $\alpha_n$ such that
\[\alpha_n-\theta
\le
\measuredangle(\dot\gamma_n(t),\bm{v}_\gamma)
\le
\alpha_n+\theta.\] 
The arc $\gamma_n$ with the maximal total curvature will solve the proposition.

It remains to choose $\delta$ so that $\theta(\delta)<\tfrac\eps{100}$.
\qeds


\parit{Proof of Proposition~\ref{prop:almost-const}.}
Set $\gamma_0=\gamma$.

Fix $\delta>0$, set $n=\lceil\tfrac2\delta\rceil$.
By Lemma~\ref{lem:eps-straight}, the geodesic $\gamma_0$ can be subdivided into $n$ arcs which are  $\delta$-straight.
Let us choose the arc $\gamma'_0$ with the maximal total curvature.
Assuming $\delta<\tfrac1{10}$ we get
\[\tc\gamma'_0\ge\tfrac\delta{10}\cdot\tc\gamma_0.\]

Let $\alpha_1$ be the angle
and $\gamma_1$ be the arc in $\gamma'_0$ 
provided by Lemma~\ref{lem:almost-const}.
In particular 
\begin{align*}
\tc\gamma_1&\ge\delta\cdot\tc\gamma'_0
\ge
\\
&\ge \tfrac{\delta^2}{10}\cdot\tc\gamma_0.
\end{align*}


If 
$\alpha_1\le \tfrac\eps2$ or $\alpha_1\ge\pi-\tfrac\eps2$ 
and $\delta$ is small enough
then statement holds for the arc $\gamma_1$ and the vector $\bm{u}=\pm\bm{v}_{\gamma'_0}$.

Otherwise let us repeat the above construction for $\gamma_1$.
Namely, apply Lemma~\ref{lem:eps-straight} to the geodesic $\gamma_1$ and
denote by $\gamma_1'$ the $\delta$-straight 
arc with maximal total curvature.
If $\delta$ is small, we get 
\[
\tfrac\eps3
<
\measuredangle(\bm{v}_{\gamma_1'},\bm{v}_{\gamma'_0})
<
\pi-\tfrac\eps3
\eqlbl{eq:v1-perp-v_2}\]
Again, we get
\[\tc\gamma_1'\ge\tfrac\delta{10}\cdot\tc\gamma_1\ge \tfrac{\delta^3}{100}\cdot\tc\gamma_0\]

Further apply Lemma~\ref{lem:almost-const} to $\gamma_1'$.
Denote by $\gamma_2$ and $\alpha_2$ the angle and the sub-arc of $\gamma_1'$.
Again
\[\tc\gamma_2\ge\tfrac{\delta^4}{100}\cdot\tc\gamma_0\]


The curve $\gamma_2$ runs under nearly constant angle to  $\bm{v}_{\gamma'_0}$ and $\bm{v}_{\gamma'_1}$.
The inequality \ref{eq:v1-perp-v_2}
makes possible to apply Lemma \ref{lem:almost-const-angles}.
Hence the main statement in the proposition follows.

Straightforward computations prove the last statement.
\qeds



\section{Drifting geodesics}\label{sec:drifting}

In this section we fix notations which will be used further 
without additional explanation.

Fix a $(x,y,z)$-coordinates on the Euclidean space;
denote by $(\bm{i},\bm{j},\bm{k})$
the standard basis.

A plane parallel to say $(y,z)$-coordinate plane will be called $(y,z)$-plane.

\begin{thm}{Definition}
A smooth curve $\gamma\:[0,\ell]\to\RR^3$ 
is called \emph{$\bm{i}$-drifting} if both ends $\gamma(0)$ and $\gamma(\ell)$ lie on the $x$-axis and 
$\measuredangle(\dot\gamma(t),\bm{i})<\tfrac1{10}$ for all $t$.
\end{thm}

\parbf{$\bm{(\lambda,\mu,\nu)}$-frame.}
Let $\Sigma$ be a convex surface 
and $\gamma\:[0,\ell]\to \Sigma$ be an $\bm{i}$-drifting minimizing geodesic 
with unit-speed parametrization.

Given $t\in [0,\ell]$, 
consider the oriented orthonormal frame $\lambda(t),\mu(t),\nu(t)$ 
such that $\nu(t)$ is the outer normal to $\Sigma$ at $\gamma(t)$,
the vector $\mu(t)$ is lies in $(y,z)$-plane 
and therefore the vector $\lambda(t)$ lies in the plane spanned by $\nu(t)$ and the $x$-axis.
We assume in addition that $\langle\lambda,\bm{i}\rangle\ge 0$.

Since $\langle\dot\gamma(t),\bm{i}\rangle>0$, 
$\nu(t)\ne\bm{i}$
and therefore the frame $(\lambda,\mu,\nu)$ is uniquely defined for any $t\in[0,\ell]$.

\parbf{Angle functions.}
Set 
\begin{align*}
\phi(t)&=\measuredangle(\bm{i},\dot\gamma(t)),&
\psi(t)&=\tfrac\pi2-\measuredangle(\bm{i},\nu(t)),&
\theta(t)&=\tfrac\pi2-\measuredangle(\mu(t),\dot\gamma(t)),
\end{align*}

From the above definitions it follows that $|\theta(t)|,|\psi(t)|\le \tfrac\pi2$ and for each $t$ there is a right spherical triangle with legs $|\theta(t)|,|\psi(t)|$ and hypotenuse $\phi(t)$.
In particular $\cos\theta\cdot\cos\psi=\cos\phi$.
Whence we get the following.

\begin{thm}{Claim}\label{clm:alpha-phi-psi}
For any $t$ we have 
\[
\phi(t)\ge |\psi(t)|\ \ \text{and}\ \ \phi(t)\ge |\theta(t)|
\]

\end{thm}

Applying Liberman's Lemma in the direction $\bm{i}$ we also get the following.

\begin{thm}{Claim}\label{clm:7.3}
If an arc $\gamma|_{[a,b]}$ lies in the dark (light) side for $\bm{i}$ then 
the angle function $\phi$ is non-decreasing (correspondingly non-increasing)
in $[a,b]$.
\end{thm}

\section{Plane sections}

Assume $\gamma$ is a curve on a smooth strongly convex surface $\Sigma$ in $\RR^3$.
Consider a plane $L$ 
passing through two points of $\gamma$, 
say $p=\gamma(a)$ and $q=\gamma(b)$ with $a<b$.
Let $L_\pm$ be a  half-planes in $L$ bounded by the line trough $p$ and $q$.
Set $\sigma_\pm=\Sigma\cap L_\pm$.

\begin{thm}{Observation}\label{obs:cut}
If $\gamma$ is a minimizing geodesic in the smooth strongly convex surface $\Sigma\subset \RR^3$ and $a$, $b$ and $\sigma_\pm$ as above then
\[\length\sigma_\pm\ge\length(\gamma|_{[a,b]}).\]
\end{thm}

To prove the observation, it is sufficient to note that $\sigma_\pm$ are  smooth convex plane curve connecting $p$ to $q$ in $\Sigma$.

Based on this observation we give couple of estimates on drifting minimizing geodesics.

Let $\gamma\:[a,b]\to\RR^3$ be a curve and $\ell$ be a line which does not pass through points of $\gamma$.
Assume $\phi\:[a,b]\to\RR$ be a continuous azimuth angle
of $\gamma$ in the cylindrical coordinates with the axis $\ell$.
If 
\[|\phi(b)-\phi(a)|\ge 2\cdot n\cdot\pi,\]
we will say that $\gamma$ \emph{goes around} the line $\ell$
at least $n$ times.

\begin{thm}{Proposition}\label{prop:around-once}
Assume $\gamma\:[0,\ell]\to \Sigma$ is an $\bm{i}$-drifting minimizing geodesic in the convex surface $\Sigma\subset \RR^3$, a sub-segment $[a,b]\subset [0,\ell]$ and the following conditions hold
\begin{enumerate}[(i)]
\item The points $\gamma(a)$ and $\gamma(b)$ lie in a half-plane with boundary line formed by the $x$-axis
and  the arc $\gamma|_{[a,b]}$ goes around the $x$-axis at least once.
\item  The $x$-coordinate of $\gamma(a)$ is larger that the $x$-coordinate of $\tfrac12\cdot(\gamma(0)+\gamma(\ell))$.
\end{enumerate}
Then  $\gamma(b)$ lies on the dark side for $\bm{i}$.
\end{thm}

\parit{Proof.} 
Let us apply Observation~\ref{obs:cut} to the plane containing $x$-axis, $\gamma(a)$ and $\gamma(b)$.


We can assume that $\gamma(0)$ is the origin of the $(x,y,z)$-coordinate system
and both points $p=\gamma(a)$ and $q=\gamma(b)$ lie in the $(x,z)$-coordinate half-plane with $x\ge 0$, denoted by $\Pi$.
We can assume that $\sigma_+\subset \Pi$.
Let $(x_p,0,z_p)$ and $(x_q,0,z_q)$ be the coordinates of $p$ and $q$.

From the assumptions we get $x_p<x_q<2\cdot x_p$.

Assume contrary,
then $\gamma(b)$ lies on the light side for $\bm{i}$.
Then from convexity of the curve $\Pi\cap \Sigma$
we get 
\[\length\sigma_+\le \sqrt{(x_q-x_p)^2+z_p^2}\]

On the other hand, since $\gamma|_{[a,b]}$ goes around $x$-axis at least once,
we get 
\[\length\gamma|_{[a,b]}\ge \sqrt{(x_q-x_p)^2+(z_p+z_q)^2}.\]

These two estimates contradict Observation~\ref{obs:cut}.
\qeds

\begin{thm}{Corollary}\label{cor:around-twice}
If $\Sigma$, $\gamma$, $\ell$, $a$ and $b$ as in the Proposition~\ref{prop:around-once} and the arc $\gamma|_{[a,b]}$ goes around the $x$-axis at least twice
then the arc $\gamma|_{[b,\ell]}$ lies on the dark side with respect to $\bm{i}$.
\end{thm}

\parit{Proof.} 
Fix $b'\in [b,\ell]$.
Note that one can find $a'\in [a,b]$ 
such that the assumptions of Proposition~\ref{prop:around-once} hold for the interval $[a',b']$. 
Applying the proposition we get the result.
\qeds 


\begin{thm}{Proposition}\label{prop:phi-psi}
Assume $\gamma\:[0,\ell]\to \Sigma$ is an $\bm{i}$-drifting minimizing geodesic 
in the convex surface $\Sigma\subset \RR^3$.
Assume that the arc $\gamma|_{[b,\ell]}$ 
lies on the dark side of $\Sigma$ with respect to $\bm{i}$.
If $b\le s<t\le \ell$ and the point $\gamma(s)$ lies in the plane $\Pi$ through $\gamma(t)$ spanned by $\nu(t)$ and $\lambda(t)$
then 
\[\phi(s)\le \psi(t).\]
\end{thm}

\parit{Proof.}
Let us apply Observation~\ref{obs:cut} to the plane $\Pi$ and $p=\gamma(s)$ and $q=\gamma(t)$.

Let $x_p$ and $x_q$ be the $x$-coordinates of $p$ and $q$.

Since $\gamma|_{[s,t]}$ lies in the dark side,
its Liberman's development $\tilde\gamma|_{[s,t]}$ 
with respect to $\bm{i}$ is concave.
In particular 
\[\length(\gamma|_{[s,t]})
=
\length(\tilde\gamma|_{[s,t]})
\ge
\tfrac{x_q-x_p}{\cos\phi(s)}.\]

On the other hand, convexity of $\sigma_+$ imply that
\[\length\sigma_+\le \tfrac{x_q-x_p}{\cos\psi(t)}.\]
It remains to apply Observation~\ref{obs:cut}.
\qeds

\section{\textit{s}-pairs}

Let $\Sigma\subset \RR^3$ 
be a strongly convex surface
and $\gamma\:[0,\ell]\to\Sigma$ be an $\bm{i}$-drifting minimizing geodesic.

We can assume that the horizon $\omega_{\bm{j}}$, 
is a smooth curve and $\gamma$ intersects them transversely.

Let $t_0<t_1<\dots<t_k$ be the meeting moments of $\gamma$ with $\omega_{\bm{j}}$.
Set
\begin{align*}
\phi_n&=\phi(t_n)
&
\psi_n&=\psi(t_n)
&
\theta_n&=\theta(t_n).
\end{align*}
Note that  $\theta_n=\pm\alpha_n$
so we can say $s_n\cdot\theta_n=(-1)^m\cdot \alpha_n$ 
for some sequence of signs $s_i=\pm1$.
In particular
\[\alpha_0-\alpha_1+\dots+(-1)^k\cdot\alpha_k
=
s_0\cdot\theta_0+s_1\cdot\theta_1+\dots+s_k\cdot\theta_k.\]
Note that for the right choice of orientation,
if $s_n=+1$ 
then $\nu_{\gamma(t)}$ moves clockwise in $\SS^2$
at $t_n$
and if $s_n=-1$ then it moves counterclockwise.    

We say that a pair of indexes $i<j$
forms an \emph{$s$-pair} 
if 
\[
\sum_{n=i}^js_n=0\ \ 
\text{and}\ \ 
\sum_{n=i}^{j'}s_n>0
\]
if $i<j'<j$.

If you exchange ``$+1$'' and ``$-1$'' in $s$ by ``$($'' and ``$)$'' correspondingly then $(i,j)$ is an $s$-pair
if and only if the $i$-th bracket forms a pair with $j$-bracket.

Note that  any index $i$ appears in at most one $s$-pair and 
for any $s$-pair $(i,j)$ we have
\begin{itemize}
\item $s_i=1$; that is, $i$-th bracket has to be opening.
 \item $s_j=-1$; that is, $j$-th bracket has to be closing.
\end{itemize}
In particular,
\begin{align*}
s_i\cdot\theta_i+s_j\cdot\theta_j&=\theta_i-\theta_j=
\\
&=(-1)^i\cdot\alpha_i+(-1)^j\cdot\alpha_j.
\end{align*}



\begin{wrapfigure}{r}{52 mm}
\begin{lpic}[t(-7 mm),b(1 mm),r(0 mm),l(0 mm)]{pics/s-pair(1)}
\lbl[br]{2,4;$+$}
\lbl[br]{3,14;$+$}
\lbl[br]{4,24;$+$}
\lbl[br]{6,34;$+$}
\lbl[bl]{16,34;$-$}
\lbl[br]{28,34;$+$}
\lbl[br]{33,44;$+$}
\lbl[bl]{43,44;$-$}
\lbl[bl]{45,34;$-$}
\lbl[bl]{46,24;$-$}
\lbl[bl]{47,14;$-$}
\lbl[bl]{47.5,4;$-$}
\lbl[lt]{4,2;$\gamma(t_i)$}
\lbl[rt]{46,2;$\gamma(t_j)$}
%\lbl[b]{4,29,79;$\dots$}
%\lbl[b]{30,38,70;$\dots$}
\end{lpic}
\end{wrapfigure}

\parbf{Tongue interpretation.}
Assume $(i,j)$ is an $s$-pair.
Note that in this case there is an arc of $\omega_{\bm{j}}$
from $\gamma(t_i)$ to $\gamma(t_j)$
with monotonic $x$-coordinate.
Moreover a disc of the tone has this arc in the boundary.

The proof can be guessed from the diagram.
It shows a lift of $\gamma$ in the universal cover of strip of $\Sigma$ between $(y,z)$-planes containing $\gamma|_{[t_i,t_j]}$;
the solid horizontal lines correspond are lifts of $\omega_{\bm{j}}$.

We say that $q$ is the depth of an $s$-pair $(i,j)$
(briefly $q=\depth_{s}(i,j)$) 
if $q$ is the maximal number such that there is $q$-long nested sequence of $s$-pairs starting with $(i,j)$.
For example the $s$-pair on the diagram has depth $5$.

More precisely, the depth of $(i,j)$ is the maximal number $q$
for which there is a sequence of $s$-pairs
$(i,j)=(i_1,j_1),(i_2,j_2),\dots,(i_q,j_q)$ such that
\[i=i_1<\dots<i_q<j_q<\dots<j_1=j.\]


Note that the $s$-pair of the same depth do not overlap;
that is if  $\depth(i,j)=\depth(i',j')$
for two distinct $s$-pairs $(i,j)$ and $(i',j')$
then either $i<j<i'<j'$ or $i'<j'<i<j$.

The following proposition follow directly from the discussion above.

\begin{thm}{Proposition}\label{prop:immersion}
Let $(i,j)$ be an $s$-pair.
Then the arcs $\gamma|_{[t_i,t_j]}$ and an arc of $\omega_{\bm{j}}$ bound an immersed disc in $\Sigma$ which lies between $(y,z)$-planes through $\gamma(t_i)$ and $\gamma(t_j)$.
Moreover the maximal multiplicity of the disc is at most $\depth_{s}(i,j)$.
\end{thm}


\begin{thm}{Corollary}\label{cor:Sq}
Let us denote by $S_q$  the subset of indexes $\{1,\dots,k\}$
which are the parts of $s$-pairs with depth $q$.
Then
\[\left|\sum_{n\in S_q}(-1)^n\cdot\alpha_n\right|
=\left|\sum_{n\in S_q}s_n\cdot\theta_n\right|
\le 4\cdot\pi\cdot q.\]
\end{thm}

\parit{Proof.} For each $n$ denote by $K_n$ the integral of Gauss curvature of the part of surface $\Sigma$ with the $x$-coordinate less that he $x$-coordinate of $\gamma(t_n)$.
Note that 
\[0\le K_1\le\dots\le K_k\le 4\cdot\pi.\]


By Proposition~\ref{prop:immersion} and the Tongue Lemma,
we get
\[s_i\cdot\theta_i+s_j\cdot\theta_j=\theta_i-\theta_j\le q\cdot (K_j-K_i)\]

The statement follows since the $s$-pairs with the same depth do not overlap.
\qeds

\begin{thm}{Corollary}\label{cor:gamma-0}
Assume 
\[q=\max_{1\le i<j\le k}\left\{\biggl|\sum_{n=i}^js_n\biggr|\right\}\]
Then
\[\left|\sum_{n=1}^k s_n\cdot\theta_n\right|
\le 2\cdot q\cdot(q+\tfrac32)\cdot \pi.
\]
\end{thm}

\parit{Proof.}
Denote by $S$ the set of all indexes which appear in some $s$-pair.

Note that depth of any $s$ pair is at most $q$.
That is,
\[S=S_1\cup\dots\cup S_q.\]
By Corollary~\ref{cor:Sq},
\[\left|\sum_{n\in S}s_n\cdot\theta_n\right|\le 2\cdot q\cdot(q+1)\cdot\pi.\eqlbl{sum-S}\]

Set $R=\{1,\dots,k\}\backslash S$;
this is the set of indexes 
which do not appear in an $s$-pair.

Given $r$, set $i\in Q_r$ 
if
\[\sum_{n=1}^is_n=r.\]
Note that $Q_r\ne\emptyset$ for at most $q$ values of $r$
and in each set $Q_r$ there are at most $2$ indexes 
which do not appear in an $s$-pair;
that is $Q_r\cap R$ has at most two indexes for each $r$.

Since $|a_n|<\tfrac\pi2$, we get
\[\left|\sum_{n\in R}s_n\cdot\theta_n\right|
\le
q\cdot\pi.
\]
The latter inequality together with \ref{sum-S} implies the statement in the corollary.
\qeds



\section{Geometric growth}\label{sec:geometric-growth}

\begin{thm}{Claim}\label{clm:alpha-psi}
Assume $\psi(t)>\eps$ for  $t\in[t_{i},t_{i+1}]$
and $s_i=s_{i+1}$
Then 
\[|\theta_{i+1}-\theta_i|>\pi\cdot\sin\eps.\] 
\end{thm}

\parit{Proof.}
Note that the arc $\gamma|_{[t_{i},t_{i+1}]}$ is a tongue with embedded disc $\iota\:\DD^2\to \Sigma$.
Since $\psi(t)>\eps$, 
the spherical image $\nu\circ\iota(\DD^2)$ 
of $\iota(\DD^2)$ 
lies in a half-disc of radius $\tfrac\pi2-\eps$ in $\SS^2$.
Note that
\[K(\iota(\DD^2))
=
\area(\nu\circ\iota(\DD^2))
<
\pi\cdot(1-\sin\eps).\]
It remains to apply Tongue Lemma~\ref{lem:tongue}.
\qeds



\begin{thm}{Claim}\label{clm:geometric-grouth}
Assume $\gamma$ lies on the dark side for $\bm{i}$.
Then for any pair of indexes $j>i$,
such that 
\[\bigl|\sum_{n=i}^{j}s_n\bigr|> 5\]
we have
\[\phi_j>\tfrac32\cdot\phi_i.\]
\end{thm}


\parit{Proof.}
By Claim~\ref{clm:7.3}, we may assume that 
\[\sum_{n=i}^{j}s_n= 6\]

Let $j'$ be the least index 
such that
\[|\sum_{n=i}^{j'} s_n|=5.\]

Note that for any $b>t_j$ there is $a\in[t_i,t_j]$
such that interval $[a,b]$ satisfies the assumptions of Proposition~\ref{prop:phi-psi}.
In particular $\psi(b)>\phi_i$ for any $b>t_j$.
Applying Claim~\ref{clm:alpha-psi},
we get that $|\theta_j|>\tfrac\pi2\cdot \phi_i$ or 
$|\theta_{j'}|>\tfrac\pi2\cdot \phi_i$.
By Claim~\ref{clm:7.3}, $\phi_n$ is non-decreasing,
and $\phi_n\ge |\theta_n|$ for any $n$,
in both cases we get
\[\phi_j>\tfrac\pi2\cdot \phi_i.\]
Hence the result follows.
\qeds



\begin{thm}{Proposition}\label{prop:graph}
If $\gamma$ is an $\bm{i}$-drifting minimizing geodesic on the dark side for $\bm{i}$ then
\[\tc_{\bm{j}}\gamma\le 100\cdot\pi.\]
\end{thm}

\parit{Proof.}
We can assume that
$\gamma$ cross the $\bm{j}$-horizon $\omega_{\bm{j}}$ transversely.
Let $t_0<\dots<t_k$ be the meeting moments of $\gamma$ with $\omega_{\bm{j}}$ and
$s_0,\dots,s_k$ the signs.

Recall that $S_q$ denotes the subset of indexes $\{1,\dots,k\}$
which appear in $s$-pair with depth $q$.
According to Corollary~\ref{cor:Sq},
\[\left|\sum_{n\in S_q}s_n\cdot\theta_n\right|\le 4\cdot q\cdot \pi.\]
In particular,
\[\left|\sum_{n\in S_1\cup\dots\cup S_5}
s_n\cdot\theta_n\right|
\le 
40\cdot\pi.\]

Set $R=\{1,\dots,k\}\backslash (S_1\cup\dots\cup S_5)$;
this is the set of indexes which appear in $s$-pairs with depth at least $6$ 
as well as those which do not appear in any $s$-pair.

According to Claim~\ref{clm:alpha-phi-psi},
\[\left|\sum_{n\in R}
s_n\cdot\theta_n\right|
\le
\sum_{n\in R}
|\theta_n|\le \sum_{n\in R}\phi_n.\] 
To estimate the last sum will use the results in Section~\ref{sec:geometric-growth}.
First let us subdivide $R$ into 5 subsets $R_1,\dots,R_5$,
by setting 
$n\in R_m$ if $m\equiv n\pmod 5$.

Given $n\in R_m$, denote by $n'$ the least index in $R_m$ which is larger $n$;
$n'$ is defined for any $n\in R_m$ except the largest one.
According to Claim~\ref{clm:geometric-grouth}, 
$\phi_{n'}>\tfrac32\cdot \phi_n$;
that is, the sequence $(\phi_n)_{n\in R_m}$ grows faster than the geometric progression with coefficient $\tfrac32$.
Since $\phi_n$ is non-decreasing in $n$,
we get 
\[\sum_{n\in R_m}\phi_n< 3\cdot\phi_k.\]
It follows that 
\[\sum_{n\in R}\phi_n< 15\cdot\phi_k\le\tfrac{15}2\cdot\pi.\]

By Corollary~\ref{cor:liberman},
\begin{align*}
\tc_{\bm{j}}\gamma
&\le 
2\cdot\pi
+
2\cdot [\alpha_0-\alpha_1+\dots+(-1)^k\cdot\alpha_k]< 
\\
&< 100\cdot \pi.
\end{align*}
\qeds



\section{Assembling of the proof}


Assume $\gamma\:[0,\ell]\to \Sigma$ is a minimizing geodesic in a convex surface $\Sigma\subset \RR^3$.

According to propositions \ref{prop:semicontinuity} and \ref{prop:convegence} we can assume that $\Sigma$ is closed,  strongly convex and smooth
and the geodesic $\gamma$ has finite length.

According to Proposition~\ref{prop:almost-const}, 
we can pass to an $\bm{i}$-drifting arc $\gamma'$
of $\gamma$ for some $(x,y,z)$-coordinate system 
such that 
\[\tc\gamma'>\tfrac{1}{100^{100}}\cdot\tc\gamma.
\eqlbl{eq:1/100}
\]
We will use the notations in Section~\ref{sec:drifting} for $\gamma'$.

Rotating $(y,z)$-coordinate plane we can ensure that
\[\tc\gamma'\le 10\cdot\tc_{\bm{j}}\gamma'\]
and that $\gamma'$ cross the horizon $\omega_{\bm{j}}$ transversally.

By Corollary \ref{cor:around-twice},
we can subdivide $\gamma'$ into at most three arcs: 
\begin{itemize}
\item \emph{Left arc} $\gamma_-'$ which lies on the light side for $\bm{i}$,
\item \emph{Middle arc} $\gamma_0'$ which rotates around $x$-axis at most $4$ times.
\item \emph{Right arc} $\gamma_+'$ which lies on the dark side for $\bm{i}$.
\end{itemize}

Indeed, choose an arc $\gamma'|_{[a,b]}$  
on the right from the $(y,z)$-plane through
$\tfrac12\cdot(\gamma'(0)+\gamma'(\ell))$
which rotates around $x$-axis 2 times 
and assume that $b$ takes the minimal possible value.  
Note that if $\gamma'(s)$ lies on $(y,z)$-plane through $\tfrac12\cdot(\gamma'(0)+\gamma'(\ell))$
then $[s,b]\supset[a,b]$ 
and any sub-arc of $[s,b]$ rotates around $x$-axis at most 2 times.

By Corollary \ref{cor:around-twice}, 
we can take $\gamma'_+=\gamma'|_{[b,\ell]}$;
in case if there is no such arc $[a,b]$, we assume that $\gamma_+'$ is not presented.
Repeat the construction reverting the direction of $x$-axis;
we get the lower arc $\gamma_-'$.
The remaining arc is assumed to be $\gamma_0'$; note that any sub-arc of $\gamma_0'$ is divided by the $(y,z)$-plane through $\tfrac12\cdot(\gamma'(0)+\gamma'(\ell))$ into two each of which rotate around $x$-axis at most 2 times. 
Therefore the number of rotations of any arc in $\gamma_0'$ is at most $4$.

Let us estimate the total curvature of $\gamma'_-$, $\gamma'_0$ and $\gamma'_+$ separately.

By Proposition~\ref{prop:graph}, we get 
\[\tc_{\bm{j}}\gamma'_+\le 100\cdot\pi.\eqlbl{+}\]
Similarly  
\[\tc_{\bm{j}}\gamma'_-\le 100\cdot\pi.\eqlbl{-}\]

By Corollary~\ref{cor:gamma-0},
\[\tc_{\bm{j}}\gamma'_0\le 100\cdot\pi.\eqlbl{0}\]
Together with \ref{+}, \ref{-} and \ref{0}
the latter implies that 
\[\tc_{\bm{j}}\gamma'\le 300\cdot\pi.\]
From \ref{eq:1/100} the result follows.
\qeds



\begin{thebibliography}{99}
\bibitem{AH-PSV}
Agarwal, P. K.; Har-Peled, S.; Sharir, M.; Varadarajan, K. R.,
``Approximating shortest paths on a convex polytope in three dimensions''.
\textit{J. ACM}
44.4 (1997),
pp. 567--584.
\bibitem{pach}
Pach, J.,
``Folding and turning along geodesics in a convex surface'',
\textit{Geombinatorics}
7.2 (1997)
pp. 61--65.
\bibitem{BKZ}
B{\'a}r{\'a}ny, I.; Kuperberg, K.; Zamfirescu, T.,
``Total curvature and spiralling shortest paths''.
\textit{Discrete Comput. Geom.}
30.2 (2003),
pp. 167--176.
\bibitem{liberman}
Либерман И. М.,
«Геодезические
 линии
 на
 выпуклых
 поверхностях»,
\textit{ДАН СССР}
32.5 (1941),
с. 310---313. 

\bibitem{usov}
Усов, В. В. 
«О длине сферического изображения геодезической на выпуклой поверхности». \textit{Сибирский математический журнал} 
17.1 (1976), 
с. 233---236.

\bibitem{berg}
Berg, I. D. 
"An estimate on the total curvature of a geodesic in Euclidean 3-space-with-boundary." 
\textit{Geometriae Dedicata} 
13.1 (1982),
pp. 1--6.

\bibitem{pogorelov}
Погорелов, А. В., 
\textit{Внешняя геометрия выпуклых поверхностей.} 
1969.

\bibitem{zalgaller}
Залгаллер, В. А. 
«Вопрос о сферическом изображении кратчайшей».
\textit{Укр. геометрический сб.}
10 (1971) 
с. 12---18.

\bibitem{milka}
Милка, А. Д. 
«Кратчайшая с неспрямляемым сферическим изображением». 
\textit{Укр. геометрический сб.} 16 (1974)
с. 35---52.

\bibitem{usov-conj-pog} 
Усов, В. В. 
«О пространственном повороте кривых на выпуклых поверхностях». 
\textit{Сибирский математический журнал}
17.6 (1976),
с. 1427---1430.

\bibitem{milka-liberman}
Милка, А. Д. 
«Аналог теоремы Либермана в римановом пространстве».
Украинский геометрический сборник,
24 (1981), 
с. 82---84,

\bibitem{petrunin}
Petrunin, A.
``Applications of quasigeodesics and gradient curves''.
\textit{Math. Sci. Res. Inst. Publ.}
30 (1997),
pp. 203--219

\bibitem{milka-bending}  Милка, А. Д. «Кратчайшие  линии  на  выпуклых  поверхностях».  Докл.  АН  СССР,   1979,   248, \textnumero1,  34---36  

\bibitem{petrunin-orthodox}
Petrunin, A.
``Exercises in Orthodox Geometry''.
\texttt{arXiv:0906.0290 [math.HO]}
\end{thebibliography}
\end{document}

\usepackage[
sorting=none,
bibencoding=auto,
backend=biber,
maxnames=10,
%babel=other,
autolang=other,
doi=false,
url=false,
style=numeric-comp,
isbn=false]{biblatex}
\usepackage{csquotes}
\AtEveryBibitem{\clearlist{language}}
\AtEveryBibitem{\clearlist{note}}
\renewbibmacro{in:}{}
%\renewcommand*{\bibfont}{\small}
\addbibresource{lib.bib}

\sloppy
\printbibliography[heading=bibintoc]
\fussy
