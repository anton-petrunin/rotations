\documentclass[a4paper,10pt]{amsart}
\usepackage{kubik}

\begin{document}
\title{Turn of geodesics on convex surfaces}
\author{Nina Lebedeva}
\address{N. Lebedeva\newline\vskip-4mm
Math. Dept.
St. Petersburg State University,
Universitetsky pr., 28, 
Stary Peterhof, 
198504, Russia.
\newline\vskip-4mm
Steklov Institute,
27 Fontanka, St. Petersburg, 
191023, Russia.}
\email{lebed@pdmi.ras.ru}
\author{Anton Petrunin}
\address{A. Petrunin\newline\vskip-4mm
Math. Dept. PSU,
University Park, PA 16802,
USA}
\email{petrunin@math.psu.edu}
\thanks{N.~Lebedeva was partially supported by RFBR grant 
14-01-00062.}
\thanks{A.~Petrunin was partially supported by NSF grant DMS 1309340.}


\date{}

\begin{abstract}
We give a universal upper bound for the variation of turn of minimizing geodesic on a convex surface in the Euclidean space.
\end{abstract}
\maketitle


\section{Introduction}

Denote by $\EE^3$ the 3-dimensional Euclidean space.

Recall that the \emph{turn} of a curve $\gamma\:\II\to \EE^3$ 
(briefly $\turn\gamma$)
is defined as supremum of sum of exterior angles 
for the broken lines inscribed in $\gamma$.
If $\gamma$ is smooth and equipped with the natural parameter, 
then 
\[\turn\gamma=\int\limits_\II \kappa(t)\cdot dt,\]
where $\kappa(t)=|\gamma''(t)|$ is the curvature of $\gamma$ at $t$.

\begin{thm}{Main theorem}\label{thm:main}
If $K$ is a closed convex set in the 3-dimensional Euclidean space,
$\Sigma$ is the surface of $K$ 
and $\gamma$ be a minimizing geodesic in $\Sigma$
then 
\[\turn\gamma\le \omega,\]
where $\omega$ is a universal real constant.
\end{thm}

The question was stated in \cite{AH-PSV}, \cite{pach} and \cite{BKZ},
but we have learned it from Dmitry Burago only few years ago.

Let us briefly discuss the related results.

\begin{itemize}
\item In \cite{liberman}, Liberman gives a bound on the turn of short geodesic in terms of the ratio diameter and inradius of $K$.
In the proof he use now so called Liberman's lemma \ref{lem:liberman} discussed below.
This statement was rediscovered in \cite{BKZ}.
\item In \cite{usov}, Usov gives the optimal bound for turn of geodesic on the graph of $\ell$-Lipscitz convex function. 
Namely, he proves that if $f\:\RR^2\to\RR$ is $\ell$-Lipschitz and convex then any 
geodesic in its graph 
\[\Gamma_f=\set{(x,y,z)\in \RR^3}{z=f(x,y)}\] 
has turn at most $2\cdot \ell$.
This statement was also rediscovered in \cite{BKZ}.
Yet an amusing generalization of Usov's result is given by Berg in \cite{berg}.
\item In \cite{pogorelov}, Pogorelov conjectured that any the spherical image of geodesic on convex surface has to be contructable.
It is easy to see that the length of spherical image of geodesic can not be smaller than its turn, so this conjecture (if true) would be stronger than Liberman's theorem.
Counterexamples were found indepenently by Milka in \cite{milka}, 
Usov in \cite{usov-conj-pog} 
and much later by Pach in \cite{pach}.
\item In \cite{BKZ},
B{\'a}r{\'a}ny,
Kuperberg, 
and Zamfirescu 
have constructed a corkscrew minimizing geodesic on a closed hypersurface;
that is a minimizing geodesic which twists around given line arbitrary many times.
\end{itemize}


\section{Preliminaries}

Let $\Sigma$ be a convex hypersurface in the Euclidean space.

Given a point $p\in \Sigma$, we will denote by $n_p$ the outer normal vector of $\Sigma$ at $p$;
the map $\Sigma\to\SS^2$ defined as $p\mapsto n_p$ is also called \emph{Gauss map}.

Fix a points $z\notin\Sigma$ and $p\in \Sigma$.
We say that $p$ lies on light (dark) side from $z$ if 
if $\langle z-p,n_p\rangle\le 0$ (correspondingly $\langle z-p,n_p\rangle\ge 0$).
If $\langle z-p,n_p\rangle= 0$ we say that $p$ lies on the horizon
%crest???
from $p$.
Note that if $z$ lies inside of $\Sigma$ then all points on $\Sigma$ lie on the dark side from $z$.

Let $\gamma$ be a space curve equipped with natural parameter.
Fix a point $z\notin\gamma$. 
Let us define \emph{Liberman's development} of $\gamma$ with respect to $z$ as the unit-speed plane cure $\tilde\gamma_z$ such that the direction $\tilde\gamma_z(t)$
changes counterclockwise as $t$ changes
and
$|\tilde\gamma_p(t)|=|\gamma(t)-z|$ for any $t$.

Locally convex/concave???


\begin{thm}{Liberman lemma}\label{lem:liberman}
Let $\Sigma$ be a convex surface in the Euclidean space 
$z\not\in\Sigma$ and $\gamma$ be a unit-speed geodesic in $\Sigma$.
Then the development $\tilde\gamma_z$ is locally convex (concave) 
at the points on dark (light) side of $\Sigma$ with respect to $z$.
\end{thm}

Assume $\gamma\:[0,\ell]\to \Sigma$ is a unit-speed curve in the space.

The vector $\gamma''(t)$ is the curvature vector of $\gamma$ at $t$.
The turn of $\gamma$ can be defined as 
\[\turn\gamma\df\int\limits_0^\ell|\gamma''(t)|\cdot dt.\]

The turn of $\tilde\gamma_z$ is called the turn of $\gamma$ in the direction of $z$ and denoted as $\turn_z\gamma$
Given a point $z$, let us define the turn of $\gamma$ in the direction of $z$ as
\[\turn_z\gamma
\df
\int\limits_0^\ell
\left|
\langle\gamma''(t),\tfrac{z-\gamma(t)}{|z-\gamma(t)|}\rangle
\right|
\cdot dt.\]
Clearly $\turn_z\gamma\le \turn \gamma$ for any $\gamma$.

\begin{thm}{Key Lemma}
\label{lem:key}
Let $\gamma\:[0,\ell]\to \Sigma$ be a geodesic on the convex surface in the Euclidean space 
and $u\in\SS^2$.
Assume that $0=t_0<t_1<\dots<t_n=\ell$ be the values such that each arcs $\gamma|_{[t_{i-1},t_i]}$ alternating light and dark side of $\Sigma$ with respect to $u$.
Set $\alpha_i=\measuredangle(\dot\gamma(t_i),u)$
Then 
\[\turn_u\gamma=|\sum_i(-1)^i\alpha_i|.\]

Moreover, if $1<i<n$ 
and $\Omega_i$ denotes the domain of $\Sigma$ bounded by the arc $\gamma|_{[t_{i-1},t_i]}$ and the $u$-horizon then 
\[|\alpha_i-\alpha_{i-1}|\le \curv\Omega_i,\]
where $\curv\Omega_i$ denotes the total curvature of $\Omega_i$.
In particular,
\[\turn_u\gamma\le 4\cdot\pi+\sum_i\curv\Omega_i.\]
\end{thm}

\parbf{Remark.}
Let $N=N(\Sigma,\gamma, u)$ be the maximal integer such that at most $N$ of the domains $\Omega_i$ intersect at one point.
Note that from \cite{BKZ}, it follows that the value $N$ can take arbitrary large value.

Then
\[\sum_{i=2}^{n-1}\curv\Omega_i
\le 
N\cdot\curv\Sigma
\le 
4\cdot N\cdot\pi.\]
Therefore, we get an estimate
\[\turn_u\gamma
\le 
4\cdot N\cdot\pi+|\alpha_0-\alpha_1|+|\alpha_{n-1}-\alpha_n|
\le
(4\cdot N+2)\cdot\pi.\]

\section{Reduction to a simpler case}

Arguing by contradiction 
let us assume that 
there is a convex surface with a minimizing geodesic which has arbitrary large turn.

In this we will show that one can assume in addition something about the surface and the curve.
Namely we will prove the following claim.

\begin{thm}{Proposition}
Assume Main Theorem does not hold;
that is, there is a sequence of convex surfaces $\Sigma_n$
and a sequence of minimizing geodesic $\gamma_n$ in $\Sigma_n$ such that 
\[\turn \gamma_n\to\infty\ \ \text{as}\ \ n\to\infty\]
Then we can assume in addition that $\Sigma_n$ is a graph on a concave function 
$z=f_n(x,y)$ and the velocity vector of $\gamma_n$
has negative $z$-coordinate at all times.
\end{thm}

\parit{Proof.}
Take a sequence of surfaces $\Sigma_n$ with a sequence of minimizing geodesics $\gamma_n$ such that 
\[\turn\gamma_n\to \infty.\]
Denote by $K_n$ the convex body bounded by $\Sigma_n$.

Applying rescaling, we can assume that $\diam\gamma_n=1$ for all $n$.
Passing to a subsequence if necessary, we can also assum that $K_n$ converges in Hausdorff sense, say to $K_\infty$.

By Liberman's theorem we have that $K_\infty$ lies in the plane.

It follows that $\gamma_n$ converges to a line segment in $K_\infty$
or broken line made from two line segments, one on one side and the other on the other side of $K_\infty$.
In the later case we can pass to an arc of $\gamma_n$ so that its turn is still converges to $\infty$ as $n\to\infty$ and the limit curve is a line segment.

We can pass to an arc of $\gamma_n$ and rescale to ensure that 
\[1\le|p_n-\gamma(t)|\le 2\] 
for any $t$ and some fixed point $p_n\in K_n$
while still $\turn\gamma_n\to\infty$ as $n\to \infty$.
In particular 
\[\limsup_{n\to\infty}(\turn_{p_n}\gamma_n)\le \pi.\]

Therefore, passing to a subarc of $\gamma_n$ we can assume that 
$\turn_{p_n}\gamma_n\to0$ and $\turn\gamma_n\to\infty$ as $n\to\infty$.

Passing to a subsequence we can assume that for all $t$, 
the angle between $\gamma'_n(t)$ and $p_n-\gamma_n(t)$ 
converges to a fixed value, 
say $\phi$ 
as $n\to \infty$.

Let us show that we can assume that $\phi=0$.
If $\phi=\pi$ then it is sufficient to change reparametrize each $\gamma_n$ in the opposite direction.
Otherwise repeat the construction for a point $q_n$ instead of $p_n$, 
so that the triangles $\triangle p_nq_n\gamma_n(t)$ are far from being degenerate.
We get that $\gamma_n$ runs in nearly one direction.
Then one can choose $p_n$ near the end of $\gamma_n$.

Now move each $\Sigma_n$ so that the segment form $p_n$ to the end of $\gamma_n$ lies on the $z$-axis,
say ...


\qeds



In the proof we will need the following lemma.

\begin{thm}{Lemma}\label{lem:1}
Let $f\:[0,a]\to\RR$ be a positive concave nondecreasing function.
Assume  $0<x_1<x_2\le a$.
Denote by $\alpha$ the the graph of the restriction $f|_{[x_1,x_2]}$.
Then $\alpha$ has smaller length than then any curve between the ends of $\alpha$ which pass through a point on $x$-axis.

In other words 
\[\length\alpha<\sqrt{(x_2-x_1)^2+(f(x_1)+f(x_2))^2}.\]

\end{thm}

The proof below is a straigthforward computaion.

\parit{Proof.}
Set $\theta= \frac{f(x_1)}{1+x_1}$.
Since $f$ is concave 
\[0\le f'(x)\le \theta.\]
In particular $f(x_2)\ge f(x_1)$ 
and 
\[\length\alpha\le \sqrt{1+\theta^2}\cdot(x_2-x_1).\]
On the other hand, 
\[\sqrt{(x_2-x_1)^2+(f(x_1)+f(x_2))^2}>\sqrt{1+\theta^2}\cdot(x_2-x_1).\]
Hence the claim follows.
\qeds

\begin{thm}{Lemma}\label{lem:2}
Let $K\subset \RR^3$ 
be an unbounded convex set which contains a ray in the direction $u$,
$\Sigma$ be the surface of $K$ 
and
$\gamma$ is a geodesic on $\Sigma$.
Assume that Liberman's development of $\gamma$ in the direction of $u$ is a line segment.
Then $\gamma$ is a line segment.
\end{thm}




\parit{Proof of the Main theorem (\ref{thm:main}).}
Assume contrary;
i.e., there is a sequence of closed convex sets $K_n$
with surfaces $\Sigma_n$ 
and geodesics $\gamma_n$ in $\Sigma_n$
such that 
\[\turn\gamma_n\to\infty
\ \ \text{as}\ \ n\to \infty.
\eqlbl{*}
\] 

Let us choose an integer $k_n$ 
(say, $k_n=\lfloor\sqrt{\turn\gamma_n}\rfloor$ will do)
such that $k_n\to\infty$ as $n\to \infty$
and if we cut $\gamma_n$ into $k_n$ arcs then we can choose one of the arcs $\gamma'_n$ for each $n$ 
so that \ref{*} holds with $\gamma'_n$ istead of $\gamma_n$.

By Liberman's lemma we can assume that turn of $\gamma_n'$ toward to one of the ends of $\gamma$ converges to zero.
After rescaling, we can assume that $\gamma_n'$ has unit length
and the distance from $\gamma'_n$ to end of $\gamma_n$ converges to infinity.

Passing to a subsequence,
we can assume that $K_n$ converges to an unbounded convex set, 
say $K_\infty$ 
and $\gamma'_n$ converges to a minimizing geodesic $\gamma'_\infty$ in the surface $\Sigma_\infty$ of $K_\infty$.
Note that $\gamma'_\infty$ satisfies the condition in Lemma \ref{lem:2};
it follows that $\gamma'_\infty$ is a line segment. 

by Liberman's theorem, $K_\infty$ is degenerate;
i.e., $K_\infty$ lies in a plane.
In other words, $K_\infty$ is either closed convex plane set or a closed line interval (possibly unbounded).
Recall that in the first case $\Sigma_\infty$ (i.e., the surface of $K_\infty$)
is defined as doubling of $K_\infty$ in its boundary;
in the later case the surface of $K_\infty$ is defined as $K_\infty$ itself.



\begin{thm}{Claim}
Without loss of generality, we can assume that $K_\infty$ is noncompact, $\gamma_\infty$ is a line segment 
and the $\gamma_n\to \gamma_\infty$ in the $C^1$-topology.
\end{thm}

If $\gamma_\infty$ is a broken line made from two line segments
then we can cut each $\gamma_n$ in two geodesics, such that each converge to a line segment.
Then we have to choose one of this halphs and pass to a susequence so that \ref{*} still holds.

It remains to show that the convergence can be assumed to be $C^1$.


We can assume that all $\gamma_n$ have unit length
and equipped with natural parametrization by $[0,1]$ and $\gamma_n(0)$ is the origin of $\RR^3$ for all $n$.
Consider the Liberman's development $\tilde\gamma_n$ of $\gamma_n$ with respect to $\gamma_n(0)=0$. 
Note that 
\[|\tilde\gamma_n(1)-\tilde\gamma_n(\tfrac12)|\le \tfrac12=\length(\gamma_n|_{[\frac12,1]}).\]
Since $\gamma_\infty$ is a line segment, we get
\[|\tilde\gamma_n(1)-\tilde\gamma(\tfrac12)|\to \tfrac12\ \ \text{as}\ \ n\to \infty.\]

By Liberman's lemma, $\tilde\gamma_n$ is a convex curve therefore for any $t\in[\tfrac12,1]$ we have
\[\measuredangle(\tilde\gamma_n(t),\tilde\gamma'_n(t))=\measuredangle(\gamma_n(t),\gamma'_n(t))\to0
\ \ \text{as}\ \ 
n\to \infty.\]
Further note that
\[
\measuredangle(\gamma_n(t),\gamma_n(1))
\le
\measuredangle(\tilde\gamma_n(t),\tilde\gamma_n(1))
\to 0 
\ \ \text{as}
\ \ 
n\to \infty.
\]
Finally
\[\measuredangle(\gamma_n(1),\gamma_\infty(1))\to 0\ \ \text{as}
\ \ 
n\to \infty.\]
Therefore 
\[\measuredangle(\gamma_n'(t),\gamma_\infty(1))\to 0\ \ \text{as}
\ \ 
n\to \infty\]
for any $t\in[\tfrac12,1]$.
In other words, on the interval $[\tfrac12,1]$, the curves $\gamma_n$ converge to the line segment $\gamma_\infty$ in the $C^1$-topology.

Repeating the same argument for the development with respect to the other end of $\gamma_n$ finishes the proof of the claim. 

\medskip

Fix a orthonormal frame $\tau,\nu,\beta$.
Let us parametrize the surfce $\Sigma_n$ by standard sphere


Fix a linear function $s$ such that $s\circ\gamma_\infty$ 
has derivative near $1$ at all points.
Fix a line $\ell$ such that $s$ is contsnt on $\ell$.

Fix large $n$.
Given a point $p\in \Sigma_n$,
the set 
\[\set{q\in\Sigma_n}{s(q)=s(p)}\]
will be called the parallel of $p$.
Typically parallel is a closed curve,
but it might degenerate to a point.
In the later case $s$ admits maximum or minimum at $p$
and we will call $p$ north or correspondingly south pole of $\Sigma_n$.
The horizon with respect to $\ell$ divides $\Sigma$ into east and west hemisphere.
If the parallel of $p$ degenerates to a point, 
then $s$ admit
Without loss of generality, we can assume that $\gamma_n$ 
cross the horizons of $\ell$ tranversally.
Let $t_0<t_1<\dots<t_k$ be the time moments 
at which $\gamma_n$ cross the horizon of $\ell$.
Let us assign the sign $s_i$ for each $t_i$;
we set $s_i=1$ or $-1$ 
if at $t_i$ the geodesic $\gamma_n$ cross the horizon from west to east and
from east to west.
The meaning of west and east can be made precise by applying normal map to $\gamma$ and equip the sphere with geographic coordinate so that the north pole will be ???. 

Given $t_k$ and an integer 
$i$ denote by $k^i$ the least integer number $m>k$ such that
\[|\sum_{j=k}^m s_j|>i\]




Note that the left and right horizons, 
say $\lambda$ and $\rho$,
on $\Sigma_n$ of $\ell$
and $\gamma_n$ go in the direction tranversal to $\Pi$.
Therefore if $\gamma_n$ meets only one of two horizons
then the domains $\Omega_i$ from Lemma \ref{lem:key}
do not overlap.
In particular we get that the turn of $\gamma_n$ is bounded above by a constant;
$6\cdot\pi$ will do, but in fact one can show that the total turn in this case converges to zero. 

The condition that $\gamma_n$ intersects only $\lambda$ or only $\rho$ always hold if $K_\infty$ is 2-dimensional.

Therefore it only remains to consider the case if $K_\infty$ is 1-dimensional.
Which we are going to do next.

\parbf{Claim.}
Without loss of generality, we can assume that $\gamma_n$ lies on the graph of convex function $z=f(x,y)$ and converges to a vertical segment as $n\to \infty$.

\medskip

Indeed ???

Consider Liberman's development $\tilde\gamma_n$ with respect to the vertical direction.
It is a graph of convex function $z=h_n(t)$;
we can assume that $h_n$ is increasing and its slope converges to infinity as $n\to\infty$.

Given $t$, denote by $t'>t$ the minimal value at which the arc $\gamma|_{[t,t']}$ makes full turn; i.e., ???.

Denote by $\tilde\alpha(t)$ the angle of $\gamma_n$ to the $z$-axis at time $t$,
it coincides 


Let us choose a direction $u$ 
which is not parallel to $\gamma_\infty$ and apply Lemma \ref{lem:key}.

For large enough $n$, 
the $\gamma_n$ 
In this case the domains $\Omega_i$ do not overlap and therefore we get an upper bound for the turn.

It remains to consider  $K_\infty$ is 1-dimensional.

Assume $\gamma$ is a minimizing geodesic on the graph of convex function $z=f(x,y)$.
Denote by $\alpha(t)$ the angle between $\gamma$ and $z$-axis at time $t$
and by $\beta(t)$ the angle between the $z$-axis and the tangent plane to the graph at $\gamma(t)$.

Clearly $\alpha(t)\ge \beta(t)$ for any $t$.

From the convexity of Liberman's development with respect to $z$-axis it follows that $\alpha(t)$ is nondecreasing function of $t$.

Assume $z$-coordinates grows as $t$ increase.
Denote by $\Pi(t)$ the vertical plane passing through $\gamma(t)$ which is perpendicular to the tangent plane.
Let $t'<t$ be a value such that $\gamma(t')\in \Pi(t)$.
Then $\beta(t)\le \alpha(t')$.

\parbf{Claim.}
$\beta(t)\le \alpha(t')$.

\section{Rotations}

Let $\gamma:[0,\ell]\to \Sigma$ be a minimizing geodesic on a convex surface $\Sigma\subset \RR^3$.

Assume that $z\:\RR^3\to\RR$ be a linear function such that
$(z\circ\gamma)'(t)>\eps$ for any $t\in (0,\ell)$ and a fixed $\eps>0$.

We can assume that $z$ is the coordinate 
of $(x,y,z)$-coordinates on $\RR^3$.
The lines parallel to the $z$-axis will be called \emph{vertical};
the lines and planes parallel to $(x,y)$-plane will be called \emph{horisontal}.

Given two values $a<b$ in $[0,\ell]$,
denote by $\rho_{[a,b]}$ the total rotation of $\gamma$ makes on the interval $[a,b]$.

Let us define $\rho_{[a,b]}$ more precisely.
If $\nu=\nu(t)$ denotes the unit normal vector to $\Sigma$ at $\gamma(t)$,
denote by $\nu^h$ its horizontal projection.
Since $(z\circ\gamma)'(t)>0$, 
we have $\nu^h(t)\ne0$ for any $t\in [0,\ell]$.
The rotation of $\rho_{[a,b]}$  is defined as
algebraic rotation around the origin of the vector $\nu^h(t)$;
say it can be defined by the formula 
\[\rho_{[a,b]}
=
\int\limits_a^b \<\mu(t),\mathrm{J}(\mu'(t))\>\cdot dt,\]
where $\mathrm{J}\:\RR^2\to\RR^2$ denotes the rotation by angle $\tfrac\pi 2$ around the origin and \[\mu\df\tfrac{\nu^h}{|\nu^h|}.\]

Given a real value $z_0$, set 
\[F_{z_0}=\set{p\in K}{z(p)=z_0}.\]

The proof of the following statement is left to the reader.

\begin{thm}{Lemma}
Let $\gamma\:[a,b]\to\Sigma$ be ???.
Set $k=\lfloor \tfrac{\rho_{[a,b]}-\pi}{2\cdot\pi}\rfloor$.
Assume a vertical line $\ell$ is passing through $F_a$ and $F_b$
and $\Pi$ be a half-plane with boundary $\ell$.
Then $\Pi$
intersects $\gamma([a,b])$ at least $k$ times
at the point say $\gamma(t_1), \dots,\gamma(t_k)$
and each arc $\gamma_{[t_i,t_{i+1}]}$ makes a full turn aroud $\ell$.
\end{thm}

Without loss of generality, we may assume that horison of $x$-axis is smooth and $\gamma$ meets the horison tranverslly.
Let $t_1<t_2<\dots<t_n$
be the values at which $\gamma$ meats the horizon.

Denote by $\alpha_i$ and $\beta_i$
the angle between $\dot\gamma(t_i)$ and $x$- and $z$-axises correspondingly.

By Lemma ???, the turn of $\gamma$ in the direction of $x$-axis
can be bounded in terms of 
\[\sigma=|\alpha_1-\alpha_2+\dots-(-1)^n\cdot\alpha_n|.\]

Note that 
\[|\alpha_i-\tfrac\pi2|\le \beta_i\]
for any $i$.



\begin{thm}{Lemma}
Assume \[z\circ\gamma(t_i)>\tfrac12\cdot(z\circ\gamma(a)+z\circ\gamma(b))\]
and $|\rho_{[t_i,t_j]}|=3\cdot \pi$ for some $j>i$.
Then the sequence $\beta_j,\beta_{j+1}\dots$ is nonecreasing.

Moreover, if for some $m>k\ge j$ we have $|\rho_{[t_k,t_m]}|=4\cdot \pi$ then 
$\beta_m\ge 2\cdot \beta_k$.
\end{thm}

Set $\#(i,j)=\rho_{[t_i,t_j]}/\pi$.
Note that $\#_{i,j}$ is an integer for any pair $(i,j)$.

Set $\sigma(i,j)=\max_{i\le i'<j'\le j}\{|\#(i',j')|\}$.
Assume $i<j$ and $\#(i,j)=0$, note that in this 
\[|\alpha_i-\alpha_j|\le \sigma(i,j)\cdot\kappa(\Sigma\cap z^{-1}([t_i,t_j])).\]

\ 
 
Let $\phi(t)=\measuredangle(\dot\gamma(t),i)$, 
$\psi(t)=\measuredangle(\dot\gamma(t),k)$ 
and $\theta(t)=\tfrac\pi2-\measuredangle(\nu(t),k)$.

Note that 
\[\theta(t)\le \psi(t)
\ \ \ 
\text{and}
\ \ \ 
\phi(t)\le \psi(t)\] 
for any $t$.


\section{Once more}

Fix a $(x,y,z)$-coordinates on the Euclidean space.
The lines parallel to the $z$-axis will be called \emph{vertical};
the lines and planes parallel to $(x,y)$-plane will be called \emph{horisontal}.

Given a vector $\upsilon$, we will denote by $\upsilon_x$, $\upsilon_y$ and $\upsilon_z$ its components.
Set 
\[e_x=(1,0,0),\ \ e_y=(0,1,0)\ \ \text{and}\ \ e_z=(0,0,1).\]

Let $\Sigma$ be a convex surface which bounds convex set $K$
and $\gamma\:\II\to \Sigma$ is a minimizing geodesic.

Given $t\in \II$, 
consider the oriented orthonormal frame $\lambda(t),\mu(t),\nu(t)$ 
such that $\nu(t)$ is the outer normal to $\Sigma$ at $\gamma(t)$,
the vector $\mu(t)$ is horizontal and therefore the vector $\lambda(t)$ lies in the plane spanned by $\nu(t)$ and the $z$-axis.

Further we assume that $\gamma'_z(t)>0$ for any $t$.
In particular, $\nu(t)$ is not vertical and therefore
the frame $(\lambda,\mu,\nu)$ is uniquely defined.

After rotating $(xy)$-plane if necessary, 
we can assume that the border of shadow in the directions of $x$- and $y$-axises 
are smooth curves and $\gamma$ intersects them transversely.

Denote by $t_1,t_2,\dots, t_k$ the time moments in the increasing order 
at which $\gamma$ intersects 
the border of shadow in the direction of $x$-axis.
Note that $\mu(t_n)=\pm e_x$;
define $s_n$ to be $\pm1$ so that
\[\mu(t_n)=s_n\cdot e_x.\]

Further denote by $\alpha(t)$ the signed angle between $\dot\gamma(t)$ and $\lambda(t)$ in the tangent plane at $\gamma(t)$;
the value $\alpha(t)$ is assumed to take values in $(-\tfrac\pi2,\tfrac\pi2)$.
Set 
\[\alpha_n=\alpha(t_n).\]

Note that  
\[\turn_{e_x}\left(\gamma|_{[t_1,t_k]}\right)
=\left|\sum s_n\cdot \alpha_n\right|.\]

Let us define two more angles.
\begin{itemize}
\item Let $\phi(t)$ be the angle between $\dot\gamma(t)$ and $z$-axis. Set $\phi_n=\phi(t_n)$.
\item Let $\psi(t)$ be the signed angle between $\nu(t)$ and $(x,y)$-plane. 
Set $\psi_n=\psi(t_n)$.
\end{itemize}
Note that 
\[\phi(t)\ge |\psi(t)|\ \ \text{and}\ \  \phi(t)\ge |\alpha(t)|\] 
for any $t$.
In particular 
\[\phi_n\ge |\psi_n|\ \ \text{and}\ \  \phi_n\ge |\alpha_n|\]
for any $n$.

Denote by $\nu^h(t)$ its horizontal projection of $\nu(t)$.
Since $\gamma'_z(t)>0$, 
we have $\nu^h(t)\ne0$ for any $t$.
The rotation $\rho_{[a,b]}$  of the interval $[a,b]$
is defined as
algebraic rotation around the origin of the vector $\nu^h(t)$;
say it can be defined by the formula 
\[\rho_{[a,b]}
=
\int\limits_a^b \<\tau(t),\mathrm{J}(\tau'(t))\>\cdot dt,\]
where $\mathrm{J}\:\RR^2\to\RR^2$ denotes the rotation by angle $\tfrac\pi 2$ around the origin and \[\tau\df\tfrac{\nu^h}{|\nu^h|}.\]

\begin{itemize}
\item $\alpha(t)$ be the signed angle between $\dot\gamma(t)$ and the plane spanned by $\nu(t)$ and $\mu(t)$.
\item $\sigma(t)$ be the continuous function
which is modulo $2\cdot\pi$ equal to the signed angle between the $y$-axis
and the plane spanned by $\nu(t)$ and $\mu(t)$.
\item Given an interval $[a,b]\subset \II$ set 
\begin{align*}
\xi[a,b]
&=
\sigma(b)-\sigma(a)
\intertext{and}
\bar\xi[a,b]
&=
\sup\set{|\sigma(b')-\sigma(a')|}{[a',b']\subset[a,b]}.
\end{align*}
\end{itemize}

Note that for any $t\in\II$,
the angles $|\phi(t)|$, $|\psi(t)|$ and $|\alpha(t)|$ form the sides of a right spherical triangle.
Therefore
\[\cos\phi=\cos\alpha\cdot\cos\psi.\]
In particular
\[|\phi|\ge |\psi|
\ \ \ 
\text{and}
\ \ \ 
|\phi|\ge |\alpha|\]
for any $t$.

\begin{thm}{Claim}
Let $[a,b]\subset\II$ and $\bar\xi[a,b]\ge 3\cdot\pi$.
Then $\psi(b)\ge \phi(a)$.
\end{thm}


\begin{thm}{Claim}
\[d\alpha= \sin \psi \cdot d\sigma\] 
\end{thm}

Without loss of generality, we may assume that horison of $x$-axis is smooth and $\gamma$ meets this horison tranverslly.
Let $t_1<t_2<\dots<t_n$
be the values at which $\gamma$ meats the horizon.
Set $\alpha_m=\alpha(t_m)$ and $s_m=\xi[t_1,t_m]$ and $e_m=e^{i\cdot s_m}$.
Note that $s_m$ is ??? of $\pi$ for each $m$,
in particular $e_m=\pm1$ for any $m$.

Note that the rotation of $\gamma$ in direction of $x$-axis equals to 
\[\sum_m (-1)^m\cdot e_m\cdot\alpha_m.\]
Let extract from the index set $1,\dots,n$
the maximal number of pairs $(i,i+1)$
such that $e_m=e_{m+1}$ note that for each pair we get

\section{The sum estimate}

There is no geometry in this section.
Here we give an estimate for a sum 
of finite sequence of real numbers 
of a very specific form.


Assume a finite  sign-sequence $\bm{s}=(s_1,\dots, s_k)$
is given;
that is $s_i=\pm1$ for $i$.

We say that a pair of indexes $i< j$
forms an \emph{$\bm{s}$-pair} 
if 
\begin{align*}
\sum_{n=i}^js_n&=0
\intertext{and }
\sum_{n=i}^{j'}s_n&>0
\end{align*} 
if $i<j'<j$.

Note that for any index $i$ appears in at most one $\bm{s}$-pair.
If you exchange ``$+1$'' and ``$-1$'' in $\bm{s}$ by ``$($'' and ``$)$'' correspondingly then $(i,j)$ is an $\bm{s}$-pair
if and only if the $i$-th bracket forms a pair with $j$-bracket;
in particular for any $\bm{s}$-pair $(i,j)$ we have
\begin{itemize}
\item $s_i=1$; that is, $i$-th braket has to be openning.
 \item $s_j=-1$; that is, $j$-th braket has to be closing.
\end{itemize}

We say that $q$ is the depth of an $\bm{s}$-pair $(i,j)$
(briefly $q=\depth_{\bm{s}}(i,j)$) 
if $q$ is the maximal number such that theis $q$-long nested sequence of $\bm{s}$-pairs starting with $(i,j)$; 
that is a sequence of $\bm{s}$-pairs
$(i,j)=(i_1,j_1),(i_2,j_2),\dots,(i_q,j_q)$ such that
\[i=i_1<\dots<i_q<j_q<\dots<j_1=j.\]

\begin{thm}{Proposition}
Assume that
\begin{itemize}
\item $s_1,\dots, s_k$ is a sign sequence,
\item $0\le K_1\le K_2\le \dots\le K_k$.
\item a sequence $\alpha_1,\dots,\alpha_k$ is such that for $\bm{s}$-pair $(i,j)$, we have
\[|\alpha_i-\alpha_j|\le \depth_{\bm{s}}(i,j)\cdot(K_j-K_i),\]
\item $0\le \beta_1\le\dots\le\beta_k$ such that $\beta_i\ge |\alpha_i|$ for any $i$ and $\beta_j>2\cdot\beta_i$ for any $j>i$ such that $|\sum_{n=i}^js_n|=5$.
\end{itemize}
Then
\[|\sum s_n\cdot \alpha_n|\le 20\cdot( K_k+ \beta_k).\]

\end{thm}

\parit{Proof.}
Note that for arbitrary the $\bm{s}$-pairs $(i,j)$ and $(i',j')$
we have three possibllities:
\begin{itemize}
\item $[i,j]\subset [i',j']$ and in this case $\depth_{\bm{s}}(i,j)<\depth_{\bm{s}}(i',j')$;
\item $[i,j]\supset [i',j']$ and in this case $\depth_{\bm{s}}(i,j)>\depth_{\bm{s}}(i',j')$;
\item $[i,j]\cap [i',j']=\emptyset$.
\end{itemize}
In partcular, if $\depth_{\bm{s}}(i,j)=\depth_{\bm{s}}(i',j')$ then the intervals $[i,j]$ and $[i',j']$ do not overlap.


Therefore if 
\[S_q=\sum_{\depth_{\bm{s}}(i,j)=q} (\alpha_i-\alpha_j)\] 
is the sum for all $\bm{s}$-pairs with depth $q$ then 
\[|S_q|\le q\cdot K_k.\]

Since $s_i=1$ and $s_j=-1$ for any $\bm{s}$-pair $(i,j)$,
we have
\[\alpha_i-\alpha_j=s_i\cdot\alpha_i+s_j\cdot\alpha_j.\]
Therefor sum  $S_1+S_2+S_3+S_4+S_5$  contains some  terms of 
$\sum_{n} s_n\cdot \alpha_n$.
Denote by $G\subset\{1,\dots,k\}$ the remaining indexes.
Let us divide $G$ into 5 groups, say $G_1,\dots,G_5$ 
so $i$ and $j$ go into the same group if 
\[\sum_{n=i}^j s_n\equiv 0\pmod 5.\]

Note that
\begin{align*}
|\sum_{n\in G_m}s_n\cdot \alpha_n|
&\le \sum_{n\in G_m}\beta_n\le
\\
&\le 2\cdot\beta_k;
\end{align*}
the last inequality follows since
$\beta_j>2\cdot \beta_i$
if $i,j\in G_m$ and $i<j$.

Summarizing
\begin{align*}
|\sum_n s_n\cdot\alpha_n|&\le |S_1|+|S_2|+|S_3|+|S_4|+|S_5|+|\sum_{n\in G} s_n\cdot\alpha_n|\le
\\
&\le 15\cdot K_k+10\cdot \beta_k.
\end{align*}
Hence the result follows.
\qeds


\begin{bibdiv}
\begin{biblist}
\bib{AH-PSV}{article}{
   author={Agarwal, Pankaj K.},
   author={Har-Peled, Sariel},
   author={Sharir, Micha},
   author={Varadarajan, Kasturi R.},
   title={Approximating shortest paths on a convex polytope in three
   dimensions},
   journal={J. ACM},
   volume={44},
   date={1997},
   number={4},
   pages={567--584},
   %issn={0004-5411},
   %review={\MR{1481315 (99c:68241)}},
   %doi={10.1145/263867.263869},
}

 \bib{BKZ}{article}{
   author={B{\'a}r{\'a}ny, Imre},
   author={Kuperberg, Krystyna},
   author={Zamfirescu, Tudor},
   title={Total curvature and spiralling shortest paths},
   note={U.S.-Hungarian Workshops on Discrete Geometry and Convexity
   (Budapest, 1999/Auburn, AL, 2000)},
   journal={Discrete Comput. Geom.},
   volume={30},
   date={2003},
   number={2},
   pages={167--176},
   %issn={0179-5376},
   %review={\MR{2007957 (2004h:52009)}},
   %doi={10.1007/s00454-003-0001-z},
}

\bib{berg}{article}{
   author={Berg, I. D.},
   title={An estimate on the total curvature of a geodesic in Euclidean
   $3$-space-with-boundary},
   journal={Geom. Dedicata},
   volume={13},
   date={1982},
   number={1},
   pages={1--6},
   %issn={0046-5755},
   %review={\MR{679213 (84d:53049)}},
   %doi={10.1007/BF00149423},
}

\bib{liberman}{article}{
   author={Liberman, J.},
   title={Geodesic lines on convex surfaces},
   journal={C. R. (Doklady) Acad. Sci. URSS (N.S.)},
   volume={32},
   date={1941},
   pages={310--313},
  % review={\MR{0010994 (6,100g)}},
}

\bib{milka}{article}{
   author={Milka, A. D.},
   title={A shortest path with nonrectifiable spherical representation. I},
%   language={Russian},
   journal={Ukrain. Geometr. Sb.},
   number={16},
   date={1974},
   pages={35--52, ii},
%   review={\MR{0385761 (52 \#6620)}},
}

\bib{pach}{article}{
   author={Pach, J{\'a}nos},
   title={Folding and turning along geodesics in a convex surface},
   journal={Geombinatorics},
   volume={7},
   date={1997},
   number={2},
   pages={61--65},
%   issn={1065-7371},
%   review={\MR{1487759 (98k:52026)}},
}

\bib{pogorelov}{book}{
   author={Pogorelov, A. V.},
   title={Vneshnyaya geometriya vypuklykh poverkhnostei},
%   language={Russian},
   publisher={Izdat. ``Nauka'', Moscow},
   date={1969},
   pages={759},
%   review={\MR{0244909 (39 \#6222)}},
}

\bib{usov}{article}{
   author={Usov, V. V.},
   title={The length of the spherical image of a geodesic on a convex
   surface},
%   language={Russian},
   journal={Sibirsk. Mat. \v Z.},
   volume={17},
   date={1976},
   number={1},
   pages={233--236. %(inside back cover)
   },
%   issn={0037-4474},
%   review={\MR{0405316 (53 \#9110)}},
}

\bib{usov-conj-pog}{article}{
   author={Usov, V. V.},
   title={The three-dimensional swerve of curves on convex surfaces},
%   language={Russian},
   journal={Sibirsk. Mat. \v Z.},
   volume={17},
   date={1976},
   number={6},
   pages={1427--1430, 1440},
%   issn={0037-4474},
%   review={\MR{0442862 (56 \#1237)}},
}

\end{biblist}
\end{bibdiv}


\end{document}