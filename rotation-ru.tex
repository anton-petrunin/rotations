\documentclass[a4paper,10pt]{article}
\usepackage{kubik-ru}

\newcommand{\nofootnote}[1]{%
\begingroup\def\thefootnote{}\footnotetext{#1}\endgroup}
\begin{document}
%\fontsize{5mm}{6mm}\selectfont
\title{О вариации поворота кратчайших\\
 на выпуклых поверхностях}
\author{Нина Лебедева и Антон Петрунин}
\nofootnote{Работа
выполнена при финансовой поддержке первого из авторов фондом РФФИ грант 14-01-00062, второго фондом NSF, грант DMS 1309340.}

\newcommand{\Addresses}{{\bigskip\footnotesize

  Нина Лебедева, \par\nopagebreak\textsc{198504, Санкт-Петербург, Старый Петергоф, Университетский пр., дом 28.}
  \par\nopagebreak
  \textsc{191023, Санкт-Петербург, наб. Фонтанки, 27.}
  \par\nopagebreak
  \textit{Электронная почта}: \texttt{lebed@pdmi.ras.ru}

\medskip

  Anton Petrunin, \par\nopagebreak\textsc{Math. Dept. PSU, University Park, PA 16802, USA}
  \par\nopagebreak
  \textit{Электронная почта}: \texttt{petrunin@math.psu.edu}
}}
\date{}
\maketitle


\begin{abstract}
Показано, что вариация поворота кратчайших
на выпуклой поверхности в трёхмерном евклидовом пространстве 
не превосходит некоторой константы.
\end{abstract}


\section{Введение}


Эта заметка отвечает на вопрос заданный на Дмитрием Бураго и
этот же вопрос мы нашли позднее в \cite{AH-PSV}, \cite{pach} и \cite{BKZ}. 

\begin{thm}{Основная теорема}\label{thm:main}
Вариация поворота кратчайшей
на выпукой поверхности в трёхмерном евклидовом пространстве $\RR^3$ 
не превосходит $1000^{1000}$.
\end{thm}

Значение $2\cdot\pi$ даёт точную оценку для аналогичной задачи на плоскости.
Есть примеры кратчайших на выпукой поверхности в $\RR^3$
с вариацией поворота строго больше $2\cdot\pi$ и
предположительно, точная оценка только немного превосходит $2\cdot\pi$.
Такой пример был построен Барань, Куперберг и Замфиреску в \cite{BKZ}. 
Вот ещё несколько связанных результатов.

\begin{itemize}
\item В \cite{liberman}, Либерман даёт оценку на вариацию поворота короткой геодезической на поверхности компактного тела $K$ через отношение диаметра $K$ и радиуса вписанного в $K$ шара.
Его доказательство использует аналог леммы \ref{lem:liberman} ниже.

\item В \cite{usov}, 
Усов даёт точную оценку на вариацию поворота геодезической на графике $\ell$-липшицевой выпуклой функции. 
А именно, он доказывает, что если $f\:\RR^2\to\RR$ является $\ell$-липшицевой и выпуклой тогда любая геодезическая на графике \[\Gamma_f=\set{(x,y,z)\in \RR^3}{z=f(x,y)}\] 
имеет вариацию поворота не больше $2\cdot \ell$.
Забавное обобщение этого результата дано Бергом в \cite{berg}.

\item В \cite{pogorelov}, Погорелов сформулировал гипотезу о том, что сферический образ геодезической на выпуклой поверхности обязан быть (локально) спрямляемым.
Легко видеть, что длина сферического образа геодезической не может быть меньше её вариации поворота.
Таким образом, эта гипотеза может рассматриваться как усиление теоремы Либермана сформуливанной выше.
Контрпримеры к различным вариантам этой гипотезы были получены Залгаллером в \cite{zalgaller},
Милкой в \cite{milka}
и Усовым в \cite{usov-conj-pog}.
Часть этих результатов была переоткрыта Пахом в \cite{pach}.

\item В \cite{BKZ},
Барань,
Куперберг
и Замфиреску построили винтовую кратчайшую на выпуклой поверхности;
то есть кратчайшую, которая обходит вокруг данной прямой произвольно большое число раз.
Ими были также переоткрыты результаты Либермана и Усова упомянутые выше.
\end{itemize}

\parbf{Идея доказательства.}
Задача сводится к случаю, когда кратчайшая $\gamma$ идёт приблизительно в одном направлении, 
скажем $\gamma$ параметризована длиной и $\dot\gamma(t)\approx\bm{i}$.

В этом случае, для оценки вариации поворота достаточно оценить интеграл
\[\int\langle\ddot\gamma(t),\bm{j}\rangle\cdot dt\] 
для вектора $\bm{j}\perp\bm{i}$.
Представьте, что поверхность освещена в направлении вектора $\bm{j}$ и, таким образом, некоторая кривая $\omega$ разделяет её на тёмную сторону и светлую.
На рисунке изображены несколько комбинаторных вариантов, как $\gamma$ может встречаться с $\omega$.

\begin{center}
 \begin{lpic}[t(2 mm),b(2 mm),r(0 mm),l(0 mm)]{pics/spiral-and-snake(1)}
\lbl[b]{93,14;$\omega$}
\end{lpic}
\end{center}

В первом случае вариация поворота оценивается через интеграл Гауссовой кривизны регионов заключённых между $\gamma$ и $\omega$.
Эта оценка следует из \emph{леммы о языке}~\ref{lem:tongue},
которая является основным моментом доказательства.

Второй случай может показаться невозможным, 
однако именно так ведёт себя винтовая кратчайшая построенная в  \cite{BKZ}.
В этом случае мы доказываем, 
что считаем вариацию поворота для каждого оборота кратчайшей и доказываем, что полученная последовательность чисел растёт быстрее чем геометрически от некоторого места в середине к обоим концам.
Оценка следует поскольку вариация поворота последних оборотов не превосходит $2\cdot\pi$.

Последний случай может рассматриваться как смесь первых двух
и его доказательство проводится смешанной техникой.

\section{Предварительные сведения}

\parbf{Вариация поворота.}
\emph{Вариация поворота}, $\tc\gamma$, кривой $\gamma\:[0,\ell]\to \RR^3$ 
определяется как точная верхняя грань сумм внешних углов
ломанных вписанных в $\gamma$.

Заметим, что вариация поворота ломанной $\sigma$ равна сумме её внешних углов.

Если $\gamma$ гладкая кривая параметризованная длиной, 
то 
\[\tc\gamma=\int\limits_0^\ell \kappa(t)\cdot dt,\]
гдя $\kappa(t)=|\ddot\gamma(t)|$ обозначает кривизну $\gamma$ в момент $t$.

\begin{thm}{Предложение}\label{prop:semicontinuity}
Предположим $\gamma_n\:\II\to\RR^3$ есть последовательность кривых, сходящаяся поточечно к кривой $\gamma_\infty\:\II\to\RR^3$.
Тогда 
\[\liminf_{n\to\infty}\tc\gamma_n\ge \tc\gamma_\infty.\]
\end{thm}


\parit{Доказательство.}
Выберем ломанную $\sigma_\infty$ вписанную в $\gamma_\infty$.
Пусть $\gamma_\infty(t_0),\zz\dots,\gamma_\infty(t_k)$
её вершины в порядке появления на $\gamma_\infty$.
Рассмотрим ломанную $\sigma_n$ вписанную в $\gamma_n$ с вершинами 
$\gamma_n(t_0),\zz\dots,\gamma_n(t_k)$.
Заметим, что
\[\tc\sigma_n\to\tc\sigma_\infty\ \ \text{при}\ \ n\to\infty.\]
По определению вариации поворота, 
\[\tc\sigma_n\le\tc\gamma_n.\]
Предложение следует, поскольку ломаная $\sigma_\infty$ 
может быть выбрана так, что 
$\tc\sigma_\infty$ произвольно близко к $\tc\gamma_\infty$.
\qeds

\parbf{Сходимость множеств.}
Для замкнутого множества $\Sigma\subset \RR^3$,
обозначим через $\dist_\Sigma$ функцию расстояния до $\Sigma$,
то есть
\[\dist_\Sigma x=\inf\set{|x-y|}{y\in\Sigma}.\]

Последовательность $\Sigma_n\subset \RR^3$ замкнутых множеств 
\emph{сходится} к замкнутому множеству $\Sigma_\infty\subset \RR^3$ 
если 
$\dist_{\Sigma_n}x\to\dist_{\Sigma_\infty}x$ для любого $x\in\RR^3$ при $n\to\infty$.


\parbf{Выпуклые поверхности.}
\emph{Выпуклая поверхность} $\Sigma$ в трёхмерном евклидовом пространстве  $\RR^3$ определяется как граница замкнутого выпуклого множества $K$ с непустой внутренностью.
Если $K$ компактно, мы говорим, что $\Sigma$ \emph{замкнутая}.

Предположим $\Sigma$ гладкая.
Если в любой точке $\Sigma$ 
её главные кривизны положительны, 
то $\Sigma$ называется \emph{сильно выпуклой}. 

\begin{thm}{Предложение}\label{prop:convegence}
Пусть $\Sigma_n$ есть последовательность выпуклых поверхностей, сходящихся к выпуклой поверхности $\Sigma_\infty$.
Тогда для любой кратчайшей $\gamma_\infty$ на $\Sigma_\infty$ 
существует последовательность кратчайших $\gamma_n$ на $\Sigma_n$, сходящаяся поточечно к $\gamma_\infty$ при $n\to \infty$.
\end{thm}

\parit{Доказательство.}
Предположим кратчайшая $\gamma_\infty\:[0,\ell]\to\Sigma_\infty$ параметризована длиной.

Выберем интервал $[a,b]\subset (0,\ell)$.
Пусть $p_\infty=\gamma_\infty(a)$ и $q_\infty=\gamma_\infty(b)$.
Выберем две последовательности точек $p_n,q_n\zz\in\Sigma_n$, сходящихся к $p_\infty$ и $q_\infty$ соответственно.

Обозначим через $\gamma_n$ кратчайшую между $p_n$ и $q_n$ на $\Sigma_n$.
Заметим, что $\gamma_n$ сходится к дуге $\gamma_\infty|_{[a,b]}$
при $n\to\infty$.

Выбирая интервал $[a,b]$ ближе и ближе к $[0,\ell]$ и переходя к диагональной подпоследовательности кратчайших получаем требуемое утверждение.
\qeds

\section{Лемма Либермана}

В этом разделе мы дадим небольшое обобщение конструкции Либермана из \cite{liberman};
см. также \cite{milka-liberman} и \cite{petrunin}.

\parbf{Развёртка.}
Пусть $\gamma\:[0,\ell]\to\RR^3$ есть кривая параметризованная длиной и точка $p$ не лежит на $\gamma$.

Предположим $\tilde\gamma_p\:[0,\ell]\to\RR^2$ есть плоская кривая параметризованная длиной и для 
некоторой точки $\tilde p$ на плоскости выполняется   
\[|\tilde p-\tilde\gamma(t)|=|p-\gamma(t)|\]
для всех $t\in[0,\ell]$.
Более того, направление из $\tilde p$ в $\tilde \gamma(t)$ монотонно крутится (по часовой или против часовой стрелке). 
Тогда $\tilde\gamma_p$ называется \emph{развёрткой} $\gamma$ относительно $p$.

\begin{center}
\begin{lpic}[t(0 mm),b(9 mm),r(0 mm),l(0 mm)]{pics/convex-concave(1)}
\lbl[tl]{23,16;$\tilde\gamma_p(a)$}
\lbl[tr]{4,16;$\tilde\gamma_p(b)$}
\lbl[tl]{14,1;$\tilde p$}
\lbl[ll]{67,18;$\tilde\gamma_p(a)$}
\lbl[rt]{45,18;$\tilde\gamma_p(b)$}
\lbl[tl]{57,1;$\tilde p$}
\lbl[t]{13,-5;Выпуклая развёртка.}
\lbl[t]{56,-5;Вогнутая развёртка.}
\end{lpic}
\end{center}

Развёртка $\tilde\gamma_p$ называется \emph{выпуклой} (\emph{вогнутой}) на интервале $[a,b]$
если дуга $\tilde\gamma_p|_{[a,b]}$
отрезает от угла $\angle \tilde p^{\tilde\gamma_p(a)}_{\tilde\gamma_p(b)}$
выпуклую ограниченную (соответственно неограниченную) область.

Мы говорим, что $\tilde\gamma_p$ \emph{локально} выпуклая (вогнутая) на интервале $[a,b]$
если любая точка $x\in [a,b]$ допускает замкнутую окрестность $[a',b']$ в $[a,b]$
такую, что $\tilde\gamma_p$ выпукла (соответственно вогнута) на интервале $[a',b']$.

Перейдя к пределу приведённой конструкции при $p$ убегающем вдоль луча в направлении единичного вектора $(-\bm{u})$, мы получим предельную плоскую кривую называемую развёрткой $\gamma$ в направлении  $\bm{u}$, она будет обозначаться $\tilde\gamma_{\bm{u}}$.

Развёртка $\tilde\gamma_{\bm{u}}$ также определяется следующими свойствами
(1) развёртка $\tilde\gamma_{\bm{u}}\:[0,\ell]\to \R^2$ параметризована длиной,
(2) для некого единичного вектора $\tilde{\bm{u}}\in\RR^2$ равенство
\[\langle \tilde{\bm{u}},\tilde\gamma_{\bm{u}}(t)\rangle
=
\langle  \bm{u},\gamma(t)\rangle\]
выполняется при всех $t\in [0,\ell]$ 
и
(3) проекция $\tilde\gamma_{\bm{u}}(t)$ на прямую перпендикулярную к $\tilde{\bm{u}}$
монотонна по $t$.

Без ущерба для общности, можно предположить, что $\tilde{\bm{u}}$ является вертикальным вектором координатной плоскости.
В этом случае $\tilde\gamma_{\bm{u}}$ выпукла (вогнута) на интервале $[a,b]$ 
если двуугольник ограниченный дугой $\tilde\gamma_{\bm{u}}|_{[a,b]}$
и отрезком $[\tilde\gamma_{\bm{u}}(a)\tilde\gamma_{\bm{u}}(b)]$
выпуклый и лежит выше (соответственно ниже) отрезка $[\tilde\gamma_{\bm{u}}(a)\tilde\gamma_{\bm{u}}(b)]$.
 
\parbf{Тёмная и светлая стороны.}
Пусть $\Sigma\subset\RR^3$ выпуклая поверхность,
$p\in\Sigma$ и $z\ne p$.

Точка $p$ лежит на \emph{тёмной}  (\emph{светлой}) стороне $\Sigma$ от $z$ 
если ни одна из точек $p+t\cdot(p-z)$ не лежит внутри  $\Sigma$ при $t>0$ (соответственно при $t<0$).
Пересечение тёмной стороны со светлой называется горизонтом точки $z$ и обозначается $\omega_z$.

Заметим, что если $z$ лежит внутри $\Sigma$, 
то все точки $\Sigma$ лежат на тёмной стороне от $z$ и горизонт $\omega_z$ пуст.

Если $\Sigma$ гладкая, мы можем определить внешнюю нормаль $\nu_p$ к $\Sigma$ в $p$.
В этом случае $p$ лежит на тёмной (светлой) стороне $\Sigma$ от $z$
тогда и только тогда, когда $\langle p-z,\nu_p\rangle\ge 0$
(соответственно $\langle p-z,\nu_p\rangle\le 0$).
Если  дополнительно $\Sigma$ замкнута и сильно выпукла, тогда горизонт любой точки вне $\Sigma$
образует гладкую замкнутую кривую на $\Sigma$.

Так же мы определяем светлую/тёмную сторону и горизонт в предельном случае,
когда $p$ убегает по лучу в направлении $(-\bm{u})$.

Последнее можно также определить напрямую следующим образом.
Точка $p\in\Sigma$ лежит на тёмной (светлой) стороне для вектора $\bm{u}$, 
если ни одна из точек $p+\bm{u}\cdot t$ не лежит внутри $\Sigma$ при $t>0$, (соответственно $t<0$).
Так же, как и раньше, пересечение светлой стороны с тёмной называется \emph{горизонтом} для $\bm{u}$ 
и обозначается через $\omega_{\bm{u}}$.

Значит, эти условия можно записать как $\langle \nu_p,\bm{u}\rangle\ge 0$, и $\langle \nu_p,\bm{u}\rangle\le 0$ в гладком случае.
Если $\Sigma$ ещё и замкнута и сильно выпукла, то $\omega_{\bm{u}}$ является замкнутой гладкой кривой.

\begin{thm}{Лемма Либерамна}\label{lem:liberman}
Пусть $\gamma$ есть геодезическая на выпуклой поверхности $\Sigma\subset \RR^3$.
Тогда для любой точки $z\notin\Sigma$ развёртка $\tilde\gamma_z$ локально выпукла (локально вогнута), если $\gamma$ лежит на тёмной (соответственно на светлой) стороне $\Sigma$ от $z$.

Аналогично для единичного вектора $\bm{u}$,
развёртка $\tilde\gamma_{\bm{u}}$ локально выпукла (локально вогнута), если $\gamma$ лежит на тёмной (соответственно на светлой) стороне $\Sigma$ для $\bm{u}$.
\end{thm}

Заметим, что для любой пространственной кривой $\gamma$, неравенство 
\[\tc\tilde\gamma_{\bm{u}}\le\tc\gamma.\]
выполняется для любого единичного вектора $\bm{u}$. 
С другой стороны, вариации поворотов нескольких развёрток дают оценку на вариацию поворота изначальной кривой.
Например, если $\bm{i},\bm{j},\bm{k}$ ортонормированный базис, тогда
\[\tc\gamma
\le
\tc\tilde\gamma_{\bm{i}}+\tc\tilde\gamma_{\bm{j}}+\tc\tilde\gamma_{\bm{k}}.\]

Если $\gamma$ лежит полностью на тёмной (или светлой) стороне для направления $\bm{u}$,
тогда по лемме Либермана получаем 
\[\tc\tilde\gamma_{\bm{u}}\le \pi.\]
Отсюда следует, что если $\gamma$ пересекает горизонты $\omega_{\bm{i}}$, $\omega_{\bm{j}}$ и $\omega_{\bm{k}}$
не более чем $N$ раз, тогда 
\begin{align*}
\tc\gamma
&\le
\tc\tilde\gamma_{\bm{i}}
+\tc\tilde\gamma_{\bm{j}}
+\tc\tilde\gamma_{\bm{k}}
\le
\\
&\le(N+1)\cdot \pi.
\end{align*}
Значит, если бы Основная Теорема не выполнялась для кратчайшей 
$\gamma$, то она должна была бы пересечь горизонты $\omega_{\bm{i}}$, $\omega_{\bm{j}}$ и $\omega_{\bm{k}}$ огромное число раз.

\section{Кривизна развёртки}\label{sec:curv-develop}

Пусть $\gamma\:[0,\ell]\to \Sigma$ геодезическая, параметризованная длиной
на 
замкнутой гладкой сильно выпуклой поверхности $\Sigma\subset\RR^3$.
Предположим, что для некоторого единичного вектора $\bm{u}$,
геодезическая $\gamma$ трансверсально проходит сквозь горизонт $\omega_{\bm{u}}$ в моменты времени 
$t_0<\dots <t_k$.
Определим $\alpha_i=\measuredangle(\dot\gamma(t_i),\bm{u})-\tfrac\pi2$ для каждого $i$.
Заметим, что $|\alpha_i|\le\tfrac\pi2$.

Значения $t_i$ и $\alpha_i$ 
будут называться соответственно \emph{моментами встреч} 
и \emph{углами встреч}
геодезической $\gamma$ с горизонтом $\omega_{\bm{u}}$.

Введём новое обозначение
\[\tc_{\bm{u}}\gamma\df\tc\tilde\gamma_{\bm{u}}.\]
Из леммы Либермана \ref{lem:liberman} получаем следующее.

\begin{thm}{Следствие}\label{cor:liberman}
Пусть $\Sigma\subset\RR^3$ гладкая сильно выпуклая поверхность,
$\gamma\:[0,\ell]\to \Sigma$ геодезическая, параметризованная длиной,
и $\bm{u}$ единичный вектор.
Предположим, что  $\gamma$ трансверсально пересекает горизонт $\omega_{\bm{u}}$,
а $t_0\zz<\dots <t_k$ и $\alpha_0,\dots,\alpha_k$ их моменты и углы встреч.
Тогда
\[\tc_{\bm{u}}\gamma
\le 3\cdot\pi
+
2\cdot\left|\alpha_0-\alpha_1
+\dots +(-1)^k\cdot\alpha_k\right|.
\]

\end{thm}

Далее, для получения верхней оценки на вариацию поворота,
мы получим оценку на сумму
\[\left|\alpha_0-\alpha_1
+\dots +(-1)^k\cdot\alpha_k\right|.\]
Нахождение такой оценки является основным моментом доказательства.


\parit{Доказательство.}
По лемме Либермана,
\[\tc_{\bm{u}}(\gamma|_{[t_{i-1},t_i]})=\pm(\alpha_{i-1}-\alpha_i),\]
здесь знак «$+$» 
если $\gamma_{[t_i,t_{i+1}]}$ лежит на тёмной стороне для  $\bm{u}$
и «$-$» если на светлой.
Суммируя, получаем
\[\tc_{\bm{u}}(\gamma|_{[t_0,t_{k}]})
=
\left|\alpha_0
-2\cdot\alpha_1
+\dots+(-1)^{k-1}\cdot2\cdot\alpha_{k-1}+(-1)^k\cdot\alpha_k\right|.\]
По лемме Либермана,
\[\tc_{\bm{u}}(\gamma|_{[0,t_0]}),\tc_{\bm{u}}(\gamma|_{[t_k,\ell]})\le \pi \]
Поскольку $\alpha_0,\alpha_k\le\tfrac\pi2$, утверждение следует.
\qeds 

Обозначим через $K_p$ Гауссову кривизну поверхности $\Sigma$ в её точке $p$.

\begin{wrapfigure}{r}{37 mm}
\begin{lpic}[t(-2 mm),b(-3 mm),r(0 mm),l(0 mm)]{pics/tongue(1)}
\lbl[t]{34,12;{$\omega_{\bm{u}}$}}
\lbl[lb]{26,24;{$\gamma$}}
\lbl[t]{1,11;{\tiny $\gamma(a)$}}
\lbl[t]{15,11;{\tiny $\gamma(b)$}}
\end{lpic}
\end{wrapfigure}
 
Предположим $a<b$ моменты встреч кратчайшей $\gamma$ с $\omega_{\bm{u}}$.
Дуга $\gamma|_{[a,b]}$ будет называться \emph{$\omega_{\bm{u}}$-языком},
если существует погруженный диск
$\iota\:\DD\looparrowright\Sigma$ 
такой, что замкнутая кривая $\iota|_{\partial D}$ 
образована дугами $\gamma|_{[a,b]}$ и $\omega_{\bm{u}}$.
В этом случае погружение $\iota$ называется \emph{диском языка}.

\begin{thm}{Лемма о языке}\label{lem:tongue}
Пусть $\bm{u}$ единичный вектор,
$\gamma\:[a,b]\to\Sigma$ кратчайшая, 
а также $\omega_{\bm{u}}$-язык с диском
$\iota\:\DD\looparrowright\Sigma$
на сильно выпуклой поверхности $\Sigma\subset\RR^3$.

Тогда
\[\int\limits_{\DD} K_{\iota(x)}\cdot d_{\iota(x)}\area_\Sigma\] 
принимает одно из значений
\[\alpha-\beta,
\  -\alpha+\beta,
\ \pi-\alpha-\beta,
 \pi+\alpha+\beta \pmod{2\cdot\pi}.
\]

В частности
\[\bigl|\alpha-\beta\bigr|
\le
\int\limits_{\DD} K_{\iota(x)}\cdot d_{\iota(x)}\area_\Sigma.
\eqlbl{eq:key2}\]
Если дополнительно потребовать, что образ $\iota(\DD)$ 
полностью лежит на тёмной стороне для $\bm{u}$, тогда
\[\tc_{}\gamma
\le 
\int\limits_{\iota(\DD)} K_p\cdot d_p\area_\Sigma.\eqlbl{eq:key3}\]
\end{thm}

\parit{Доказательство.}
Поскольку $\gamma$ геодезическая, 
параллельный перенос вдоль $\gamma$ 
переводит $\dot\gamma(a)$ в $\dot\gamma(b)$.

Заметим также, что $\bm{u}$ лежит в касательной плоскости $\Sigma$ в любой точке горизонта $\omega_{\bm{u}}$.
В частности $\bm{u}$ образует параллельное поле на $\omega_{\bm{u}}$.

Отсюда следует что параллельное перенесение вокруг $\iota|_{\partial\DD}$ 
поворачивает касательную плоскость на угол 
\[\pm(\tfrac\pi2+\alpha)\pm(\tfrac\pi2+\beta)\]
Для доказательства основного утверждения леммы остаётся применить формулу Гаусса --- Бонне.

Обозначим через $R$ правую часть \ref{eq:key2}.
Заметим, что $R\ge 0$ 
и $|\alpha|,|\beta|\le \tfrac\pi2$.
Из основного утверждения леммы следует, что  
\[R\ge\bigl|\alpha-\beta\bigr|.\]

Для доказательства \ref{eq:key3}, заметим, что в этом случае $\iota$ является вложением.
Далее заметим, что сферический образ тёмной стороны $\Sigma$ образует полусферу.
Таким образом $2\cdot\pi$ есть интеграл Гауссовой кривизны по всей тёмной стороне.
Отсюда
\[\int\limits_{\DD} K_{\iota(x)}\cdot d_{\iota(x)}\area_\Sigma
=
\int\limits_{\iota(\DD)} K_p\cdot d_p\area_\Sigma
<2\cdot\pi.\]
Применение леммы Либермана завершает доказательство.
\qeds

\section{Почти прямые кривые}

Пусть $\eps>0$.
Кривая $\gamma\:[a,b]\to\RR^3$ называется \emph{$\eps$-прямой}
если
\[(1-\eps)\cdot\length \gamma\le |\gamma(b)-\gamma(a)|.\]

\begin{thm}{Лемма}\label{lem:eps-straight}
Пусть $\eps>0$ 
и $n$ положительное целое такое, что 
$n\cdot\eps>2$.
Тогда любую кратчайшую на выпуклой поверхности $\Sigma$ в $\RR^3$
можно разбить на $\eps$-прямые дуги $\gamma_1,\dots,\gamma_n$.
\end{thm}

\parit{Доказательство.}
Пусть $\theta\in(0,\pi)$
есть число определённое равенством
\[1-\cos\theta=\eps.\]

Пусть $p,q\in \Sigma$ и $\nu_p$ и $\nu_q$ --- внешние нормали к $\Sigma$ в точках $p$ и $q$ соответственно.
Заметим, что если 
\[\measuredangle(\nu_p,\nu_q)\le 2\cdot\theta,\]
то любая кратчайшая от $p$ до $q$ в $\Sigma$
является $\eps$-прямой.

Пусть $\gamma\:[0,\ell]\to \Sigma$ кратчайшая, параметризованная длиной.

Предположим, $\gamma_{[t,\ell]}$ не $\eps$-прямая,
тогда обозначим через $t'$ максимальное число в $[t,\ell)$ такое, что дуга $\gamma_{[t,t']}$ является $\eps$-прямой.
Рассмотрим последовательность $0=t_0<t_1<\dots<t_n<\ell$ такую, что $t_{i+1}=t_i'$ для каждого $i$.
Обозначим через $\nu_i$ единичную внешнюю нормаль к $\Sigma$ в $\gamma(t_i)$. 
Из вышесказанного следует, что
\[\measuredangle(\nu_i,\nu_j)\ge2\cdot\theta\] для всех $i\ne j$.
Другими словами, открытые шары $\mathrm{B}_\theta(\nu_i)$ в сфере 
$\SS^2$ не пересекаются.

Остаётся заметить, что  
\[\area[\mathrm{B}_\theta(\nu_i)] =2\cdot\pi\cdot\eps
 \ \ \text{и}\ \ 
 \area\SS^2=4\cdot\pi.
\]
Отсюда результат.
\qeds

\begin{thm}{Следствие}\label{cor:length-diameter}
Для любой кратчайшей $\gamma$ длины $\ell$ на выпуклой поверхности $\Sigma$ в $\RR^3$ выполняется неравенство 
\[\diam \gamma\ge \tfrac\ell{10}.\]

\end{thm}

\parit{Доказательство.}
Применим лемму~\ref{lem:eps-straight} для $\eps=\tfrac12$.
\qeds


\section{Дуга в почти одном направлении}

\begin{thm}{Предложение}\label{prop:almost-const}
Для любого $\eps>0$ существует $\delta>0$ такое, что выполняется следующее условие.

Если $\gamma\:[a,b]\to\Sigma$ кратчайшая 
на гладкой сильно выпуклой поверхности $\Sigma$ в $\RR^3$,
то существует подынтервал  $[a',b']\subset[a,b]$
такой, что
\[\tc(\gamma|_{[a',b']})>\delta\cdot\tc\gamma.\]
и
\[\measuredangle(\dot\gamma(t),\bm{u})<\eps\] 
для некоторого единичного вектора $\bm{u}$
и всех $t\in[a',b']$.

Более того, при $\eps=\tfrac1{10}$ можно взять $\delta=\tfrac1{100^{100}}$.
\end{thm}

Предпошлём доказательству пару лемм.

\begin{thm}{Лемма}\label{lem:almost-const-angles}
Для любого $\eps>0$ существует $\delta>0$ такое, что выполняется следующее условие.

Предположим $\gamma$ есть кривая,
$\bm{v}_1$ и $\bm{v}_2$ два ненулевых вектора в $\RR^3$
и $0\le\alpha_1,\alpha_2\le\pi$ такие, что
\begin{align*}
\eps
&<\measuredangle(\bm{v}_1,\bm{v}_2)<\pi-\eps
\\
\alpha_i-\delta
&<\measuredangle(\bm{v}_i,\dot\gamma(t))< \alpha_i+\delta,
\end{align*}
тогда существует вектор $\bm{u}$ такой, что
$\measuredangle(\bm{u},\dot\gamma(t))<\eps$.

Более того, при $\eps<\tfrac1{10}$ 
можно взять $\delta=\eps^{10}$.
\end{thm}

Доказательство даётся прямым вычислением и мы его опускаем.

\begin{thm}{Лемма}\label{lem:almost-const}
Для любого $\eps>0$ существует $\delta>0$ такое, что выполняется следующее условие.

Предположим $\gamma\:[a,b]\to\Sigma$ является $\delta$-прямой кратчайшей на гладкой сильно выпуклой поверхности $\Sigma$ in $\RR^3$.
Пусть $\bm{v}_\gamma=\gamma(b)-\gamma(a)$.
Тогда существует подынтервал $[a',b']$ в $[a,b]$
такой, что
\[\tc(\gamma|_{[a',b']})\ge\delta\cdot\tc\gamma.\]
и
\[\alpha-\eps
\le
\measuredangle(\dot\gamma(t),\bm{v}_\gamma)
\le\alpha+\eps\] 
для некоторого фиксированного $\alpha$
и
любого $t\in[a',b']$.

Более того, при $\eps<\tfrac1{10}$ можно взять $\delta=\eps^{10}$.
\end{thm}

\parit{Доказательство.}
Без ущерба для общности, можно предположить, что
$a\zz=0$, $b=2$ и 
\[\tc(\gamma|_{[1,2]})
\ge
\tfrac12\cdot\tc\gamma.\]

Пусть $p=\gamma(0)$
и $\theta\in(0,\pi)$ такое, что $1-\cos\theta=\delta$.
Заметим, что
$$\measuredangle(\bm{v}_\gamma,\gamma(t)-p)
\le
\measuredangle(\tilde\gamma_p(1)-\tilde p,\tilde\gamma_p(2)-\tilde p)
<
\theta
\eqlbl{eq:angle1}$$
для любого $t\ge 1$.

По лемме Либермана
\[\tc_p(\gamma|_{[1,2]})< \pi+\theta.\]
Предположим $N=\lceil\tfrac\pi\theta+1\rceil$.
Тогда можно подразбить $\gamma|_{[1,2]}$ на $N$ дуг 
 $\gamma_1,\gamma_2,\zz\dots,\gamma_N$ таких, что
\[\tc_p(\gamma_n)\le \theta\eqlbl{eq:tcp}\]
для каждого $n$.

Из \ref{eq:angle1} и \ref{eq:tcp},
следует, что для каждого $n$, существует угол $\alpha_n$, такой что
\[\alpha_n-\theta
\le
\measuredangle(\dot\gamma_n(t),\bm{v}_\gamma)
\le
\alpha_n+\theta.\] 
Для доказательства основного утверждения остаётся выбрать $\delta$ так, чтобы $\theta(\delta)<\tfrac\eps{100}$
и выбрать дугу $\gamma_n$ с максимальной вариацией поворота.
\qeds


\parit{Доказательство предложения~\ref{prop:almost-const}.}
Положим $\gamma_0=\gamma$.

Для $\delta>0$, положим $n=\lceil\tfrac2\delta\rceil$.
По лемме~\ref{lem:eps-straight}, геодезическую $\gamma_0$ можно подразбить на $n$ дуг, каждая из которых $\delta$-прямая.
Выберем из них дугу $\gamma'_0$ с максимальной вариацией поворота.
Предполагая $\delta<\tfrac1{10}$, получаем
\[\tc\gamma'_0\ge\tfrac\delta{10}\cdot\tc\gamma_0.\]

Пусть $\alpha_1$ есть угол 
и $\gamma_1$ --- дуга в $\gamma'_0$, 
предоставленные леммой~\ref{lem:almost-const}.
В частности,
\begin{align*}
\tc\gamma_1&\ge\delta\cdot\tc\gamma'_0
\ge
\\
&\ge \tfrac{\delta^2}{10}\cdot\tc\gamma_0.
\end{align*}


Если
$\alpha_1\le \tfrac\eps2$ (или $\alpha_1\ge\pi-\tfrac\eps2$)
и $\delta$ достаточно мало,
тогда утверждение выполняется для дуги $\gamma_1$ и вектора $\bm{u}=\pm\bm{v}_{\gamma'_0}$.

В противном случае повторим построение для $\gamma_1$.
А именно, применим лемму~\ref{lem:eps-straight} к геодезической $\gamma_1$ и
обозначим через $\gamma_1'$ ту $\delta$-прямую дугу у которой максимальна вариация поворота.
Если значение $\delta$ мало, мы получаем 
\[
\tfrac\eps3
<
\measuredangle(\bm{v}_{\gamma_1'},\bm{v}_{\gamma'_0})
<
\pi-\tfrac\eps3
\eqlbl{eq:v1-perp-v_2}\]
Таким образом
\[\tc\gamma_1'\ge\tfrac\delta{10}\cdot\tc\gamma_1\ge \tfrac{\delta^3}{100}\cdot\tc\gamma_0.\]

Далее применим лемму~\ref{lem:almost-const} к $\gamma_1'$.
Обозначим через $\gamma_2$ и $\alpha_2$ полученные угол и дугу из $\gamma_1'$.
Снова
\[\tc\gamma_2\ge\tfrac{\delta^4}{100}\cdot\tc\gamma_0.\]


Кривая $\gamma_2$ бежит под почти постоянным углом к $\bm{v}_{\gamma'_0}$ и $\bm{v}_{\gamma'_1}$.
Неравенство \ref{eq:v1-perp-v_2}
позволяет применить лемму~\ref{lem:almost-const-angles}.
Отсюда следует основная часть.

Последняя часть доказывается прямым вычислением.
\qeds



\section{Дрейфующие геодезические}\label{sec:drifting}

В этом разделе мы вводим обозначения, используемые далее.

Выберем $(x,y,z)$-координаты в евклидовом пространстве;
обозначим через $(\bm{i},\bm{j},\bm{k})$ стандартный базис.

Плоскость параллельная, скажем $(y,z)$-координатной плоскости, будет называться $(y,z)$-плоскость.

\begin{thm}{Определение}
Гладкая кривая $\gamma$ 
называется \emph{$\bm{i}$-дрейфующей}, если оба её конца лежат на оси $x$ и 
$\measuredangle(\dot\gamma(t),\bm{i})<\tfrac1{10}$ 
для любого $t$.
\end{thm}

\parbf{$\bm{(\lambda,\mu,\nu)}$-репер.}
Пусть $\Sigma$ --- гладкая выпуклая поверхность 
и $\gamma\:[0,\ell]\to \Sigma$ --- $\bm{i}$-дрейфующая кратчайшая, 
параметризованная длиной.

Для $t\in [0,\ell]$, 
рассмотрим ориентированный ортонормированный репер $\lambda(t),\mu(t),\nu(t)$ 
такой, что вектор $\nu(t)$ есть внешняя нормаль к $\Sigma$ в $\gamma(t)$,
вектор $\mu(t)$ лежит в $(y,z)$-плоскости 
и значит вектор $\lambda(t)$ лежит в плоскости параллельной $\nu(t)$ и оси $x$.
Мы дополнительно предполагаем, что $\langle\lambda,\bm{i}\rangle\ge 0$.

Поскольку $\langle\dot\gamma(t),\bm{i}\rangle>0$, имеем 
$\nu(t)\ne\bm{i}$. 
Значит репер $(\lambda,\mu,\nu)$ определён однозначно для всех $t\in[0,\ell]$.

\parbf{Угловые функции.}
Положим
\begin{align*}
\phi(t)&=\measuredangle(\bm{i},\dot\gamma(t)),&
\psi(t)&=\tfrac\pi2-\measuredangle(\bm{i},\nu(t)),&
\theta(t)&=\tfrac\pi2-\measuredangle(\mu(t),\dot\gamma(t)).
\end{align*}
Заметим, что $|\theta(t)|,|\psi(t)|\le \tfrac\pi2$ 
и для любого $t$ существует прямоугольный сферический треугольник с катетами $|\theta(t)|,|\psi(t)|$ и гипотенузой $\phi(t)$.
В частности $\cos\theta\cdot\cos\psi=\cos\phi$.
Отсюда получаем следующее.

\begin{thm}{Утверждение}\label{clm:alpha-phi-psi}
Для любого $t$ выполняются неравенства 
\[
\phi(t)\ge |\psi(t)|\ \ \text{и}\ \ \phi(t)\ge |\theta(t)|
\]

\end{thm}

Применив лемму Либермана в направлении $\bm{i}$ получаем следующее.

\begin{thm}{Утверждение}\label{clm:7.3}
Если дуга $\gamma|_{[a,b]}$ лежит на тёмной (светлой) стороне для $\bm{i}$, 
то угловая функция $\phi$ не убывает (соответственно не возрастает) на интервале $[a,b]$.
\end{thm}

\section{Сечения}

Пусть $\gamma$ --- кривая на гладкой сильно выпуклой поверхности $\Sigma$ в $\RR^3$.
Рассмотрим плоскость $L$, 
проходящую через две точки $\gamma$, 
скажем $p=\gamma(a)$ и $q=\gamma(b)$, где $a<b$.
Пусть $L_\pm$ --- полуплоскости в $L$ ограниченные прямой, проходящей через $p$ и $q$.
Положим $\sigma_\pm=\Sigma\cap L_\pm$.

\begin{thm}{Наблюдение}\label{obs:cut}
Если $\gamma$ кратчайшая на гладкой сильно выпуклой поверхности $\Sigma\subset \RR^3$, и $a$, $b$, $\sigma_\pm$ как выше, то
\[\length\sigma_\pm\ge\length(\gamma|_{[a,b]}).\]
\end{thm}

Для доказательства достаточно заметить, что $\sigma_{+}$ и $\sigma_{-}$  суть гладкие выпуклые плоские кривые, 
соединяющие $p$ с $q$ в $\Sigma$.

Из этого наблюдения мы выведем пару оценок для дрейфующих кратчайших.

Пусть $\gamma\:[a,b]\to\RR^3$ --- кривая и $\ell$ --- прямая, не проходящая через точки на $\gamma$.
Рассмотрим непрерывную функцию $\phi\:[a,b]\to\RR$ равную азимуту $\gamma$ в цилиндрических координатах с осью $\ell$.
Если
\[|\phi(b)-\phi(a)|\ge 2\cdot n\cdot\pi,\]
мы говорим, что  $\gamma$ \emph{обошла вокруг} прямой $\ell$ хотя бы $n$ раз.

\begin{thm}{Предложение}\label{prop:around-once}
Пусть $\gamma\:[0,\ell]\to \Sigma$ есть $\bm{i}$-дрейфующая кратчайшая на выпуклой поверхности $\Sigma\subset \RR^3$. Предположим, для подынтервала $[a,b]$ в $[0,\ell]$ выполняются следующие условия:
\begin{enumerate}[(i)]
\item точки $\gamma(a)$ и $\gamma(b)$ лежат в полуплоскости ограниченной осью $x$ 
и дуга $\gamma|_{[a,b]}$ обходит вокруг оси $x$ хотя бы раз.
\item  у точки $\gamma(a)$, $x$-координата больше чем у точки $\tfrac12\cdot(\gamma(0)+\gamma(\ell))$.
\end{enumerate}
Тогда  $\gamma(b)$ лежит на тёмной стороне для $\bm{i}$.
\end{thm}

\parit{Доказательство.} 
Применим наблюдение~\ref{obs:cut} к плоскости, содержащей ось $x$, точки $\gamma(a)$ и $\gamma(b)$.


Можно предположить, что $\gamma(0)$ совпадает с началом координат 
и обе точки $p=\gamma(a)$ и $q=\gamma(b)$ лежат на $(x,z)$-координатной полуплоскости с $x\ge 0$, далее обозначаемой $\Pi$.
Мы можем также предположить, что $\sigma_+\subset \Pi$.
Обозначим через $(x_p,0,z_p)$ и $(x_q,0,z_q)$ координаты $p$ и $q$.

По условиям, $x_p<x_q<2\cdot x_p$.

Рассуждая от противного, предположим, что $\gamma(b)$ лежит на светлой стороне для $\bm{i}$.
Тогда из выпуклости кривой $\Pi\cap \Sigma$
следует, что 
\[\length\sigma_+\le \sqrt{(x_q-x_p)^2+z_p^2}.\]

С другой стороны, поскольку $\gamma|_{[a,b]}$ обходит вокруг оси $x$ хотя бы раз,
получаем 
\[\length\gamma|_{[a,b]}\ge \sqrt{(x_q-x_p)^2+(z_p+z_q)^2}.\]

Эти две оценки противоречат наблюдению~\ref{obs:cut}.
\qeds

\begin{thm}{Следствие}\label{cor:around-twice}
Если $\Sigma$, $\gamma$, $\ell$, $a$ и $b$ как в предложении~\ref{prop:around-once} 
и дуга $\gamma|_{[a,b]}$ обходит вокруг оси $x$ хотя бы дважды,
то дуга $\gamma|_{[b,\ell]}$ лежит на тёмной стороне для $\bm{i}$.
\end{thm}

\parit{Доказательство.} 
Заметим, что для любого $b'\in [b,\ell]$
существует $a'\in [a,b]$ 
такое, что условия Предложения~\ref{prop:around-once} выполняются для интервала $[a',b']$. 
Остаётся применить это предложение.
\qeds 


\begin{thm}{Предложение}\label{prop:phi-psi}
Пусть $\gamma\:[0,\ell]\to \Sigma$ есть $\bm{i}$-дрейфующая кратчайшая на выпуклой поверхности $\Sigma\subset \RR^3$.
Предположим, что дуга $\gamma|_{[b,\ell]}$ 
лежит на тёмной стороне $\Sigma$ для $\bm{i}$.
Если $b\le s<t\le \ell$ и точка $\gamma(s)$ лежит в плоскости $\Pi$, проведённой через $\gamma(t)$ 
в направлении $\nu(t)$ и $\lambda(t)$,
тогда
\[\phi(s)\le \psi(t).\]
\end{thm}

\parit{Доказательство.}
Применим наблюдение~\ref{obs:cut} к плоскости $\Pi$, $p=\gamma(s)$ и $q\zz=\gamma(t)$.

Пусть $x_p$ и $x_q$ обозначают $x$-координаты $p$ и $q$.

Поскольку $\gamma|_{[s,t]}$ лежит на тёмной стороне,
её развёртка $\tilde\gamma|_{[s,t]}$ 
по отношению к $\bm{i}$ выпуклая.
В частности
\[\length(\gamma|_{[s,t]})
=
\length(\tilde\gamma|_{[s,t]})
\ge
\tfrac{x_q-x_p}{\cos\phi(s)}.\]

С другой стороны, выпуклость $\sigma_+$ влечёт, что
\[\length\sigma_+\le \tfrac{x_q-x_p}{\cos\psi(t)}.\]
Остаётся применить наблюдение~\ref{obs:cut}.
\qeds

\section{\textit{s}-пары}

Пусть  $\gamma\:[0,\ell]\to\Sigma$ --- $\bm{i}$-дрейфующая кратчайшая
на сильно выпуклой поверхности $\Sigma\subset \RR^3$.

Предположим, что $\gamma$ пересекает горизонт $\omega_{\bm{j}}$  трансверсально.

Обозначим через $t_0<t_1<\dots<t_k$ моменты встреч $\gamma$ с $\omega_{\bm{j}}$.
Положим
\begin{align*}
\phi_n&=\phi(t_n)
&
\psi_n&=\psi(t_n)
&
\theta_n&=\theta(t_n).
\end{align*}
Заметим, что $\theta_n=\pm\alpha_n$,
то есть $s_n\cdot\theta_n=(-1)^m\cdot \alpha_n$ 
для некоторой последовательности знаков $s_i=\pm1$.
В частности
\[\alpha_0-\alpha_1+\dots+(-1)^k\cdot\alpha_k
=
s_0\cdot\theta_0+s_1\cdot\theta_1+\dots+s_k\cdot\theta_k.\]
Если ориентация выбрана правильно, то  $s_n=+1$, 
когда $\nu_{\gamma(t)}$ двигается по часовой стрелке в $\SS^2$
в момент $t_n$
и если $s_n=-1$, тогда он двигается против часовой стрелки.    

Пара индексов $i<j$
образует \emph{$s$-пару}, 
если
\[
\sum_{n=i}^js_n=0\ \ 
\text{и}\ \ 
\sum_{n=i}^{j'}s_n>0,
\]
если $i<j'<j$.

Поменяв ``$+1$'' и ``$-1$'' в последовательности $s$ на скобки ``$($'' и ``$)$'' соответственно получаем, что $(i,j)$ образует $s$-пару
тогда и только тогда, когда $i$-ая скобка парна с $j$-ой скобкой.

Заметим, что каждый индекс $i$ появляется не более чем в одной $s$-паре и для любой $s$-пары $(i,j)$ имеем
\begin{itemize}
\item $s_i=1$; то есть, $i$-ая скобка должна быть открывающей.
 \item $s_j=-1$; то есть, $j$-ая скобка должна быть закрывающей.
\end{itemize}
В частности, для любой $s$-пары $(i,j)$ имеем
\begin{align*}
s_i\cdot\theta_i+s_j\cdot\theta_j&=\theta_i-\theta_j=
\\
&=(-1)^i\cdot\alpha_i+(-1)^j\cdot\alpha_j.
\end{align*}



\begin{wrapfigure}{r}{52 mm}
\begin{lpic}[t(-7 mm),b(1 mm),r(0 mm),l(0 mm)]{pics/s-pair(1)}
\lbl[br]{2,4;$+$}
\lbl[br]{3,14;$+$}
\lbl[br]{4,24;$+$}
\lbl[br]{6,34;$+$}
\lbl[bl]{16,34;$-$}
\lbl[br]{28,34;$+$}
\lbl[br]{33,44;$+$}
\lbl[bl]{43,44;$-$}
\lbl[bl]{45,34;$-$}
\lbl[bl]{46,24;$-$}
\lbl[bl]{47,14;$-$}
\lbl[bl]{47.5,4;$-$}
\lbl[lt]{4,2;$\gamma(t_i)$}
\lbl[rt]{46,2;$\gamma(t_j)$}
\end{lpic}
\end{wrapfigure}

\parbf{$\bm{s}$-пары и $\omega_{\bm{j}}$-языки.}
Пусть $(i,j)$ --- $s$-пара.
Заметим, что существует дуга в $\omega_{\bm{j}}$ от $\gamma(t_i)$ к $\gamma(t_j)$
с монотонной $x$-координатой.
Более того, существует $\omega_{\bm{j}}$-язык с этой дугой, включённой в граничную кривую.

Доказательство показано на рисунке.
Там изображено поднятие $\gamma$ в универсальное накрытие полоски в $\Sigma$ между $(y,z)$-плоскостями через $\gamma(t_i)$ и $\gamma(t_j)$.
Горизонтальные линии соответствуют поднятиям $\omega_{\bm{j}}$.

Положительное целое число $q$
назовём \emph{глубиной} $s$-пары $(i,j)$ 
(или $q\zz=\depth_{s}(i,j)$), 
если $q$ максимальное число для которого существует \emph{убывающая} последовательность $s$-пар начинающаяся $(i,j)$.
Например $s$-пара на рисунке имеет глубину $5$.

Точнее, глубина $(i,j)$ равна максимальному числу $q$
для которого существует последовательность $s$-пар,
$(i,j)=(i_1,j_1),(i_2,j_2),\zz\dots,(i_q,j_q)$ такая, что
\[i=i_1<\dots<i_q<j_q<\dots<j_1=j.\]


Заметим, что $s$-пары с одной и той же глубиной не перекрываются.
То есть, если $\depth(i,j)=\depth(i',j')$
для различных $s$-пар $(i,j)$ и $(i',j')$,
тогда $i\zz<j\zz<i'\zz<j'$ либо $i'<j'<i<j$.

Отсюда получаем следующее предложение.

\begin{thm}{Предложение}\label{prop:immersion}
Пусть $(i,j)$ --- $s$-пара.
Тогда дуги $\gamma|_{[t_i,t_j]}$ и некоторая дуга в $\omega_{\bm{j}}$ ограничивают погруженный диск в $\Sigma$, лежащий между $(y,z)$-плоскостями через $\gamma(t_i)$ и $\gamma(t_j)$.
Более того, максимальная кратность погружения не превосходит $\depth_{s}(i,j)$.
\end{thm}


\begin{thm}{Следствие}\label{cor:Sq}
Обозначим через $S_q$ подмножество индексов $\{1,\dots,k\}$
которые встречаются в $s$-парах глубины $q$.
Тогда
\[\left|\sum_{n\in S_q}(-1)^n\cdot\alpha_n\right|
=\left|\sum_{n\in S_q}s_n\cdot\theta_n\right|
\le 4\cdot\pi\cdot q.\]
\end{thm}

\parit{Доказательство.} Для индекса $n$ обозначим через $K_n$ интеграл Гауссовой кривизны куска поверхности $\Sigma$ с $x$-координатой меньше чем $x$-координата $\gamma(t_n)$.
Заметим, что
\[0\le K_1\le\dots\le K_k\le 4\cdot\pi.\]


По предложению~\ref{prop:immersion} и лемме о языке, для $s$-пары $(i,j)$ глубины $q$,
получаем
\[s_i\cdot\theta_i+s_j\cdot\theta_j=\theta_i-\theta_j\le q\cdot (K_j-K_i).\]

Утверждение следует, поскольку $s$-пары с одной и той же глубиной не перекрываются.
\qeds

\begin{thm}{Следствие}\label{cor:gamma-0}
Предположим, 
\[q=\max_{1\le i<j\le k}\left\{\biggl|\sum_{n=i}^js_n\biggr|\right\}\]
Тогда
\[\left|\sum_{n=1}^k s_n\cdot\theta_n\right|
\le 2\cdot q\cdot(q+\tfrac32)\cdot \pi.
\]
\end{thm}

\parit{Доказательство.}
Обозначим через $S$ множество всех индексов, которые появляются в $s$-парах.

Заметим, что глубина $s$-пар не превосходит $q$.
Таким образом
\[S=S_1\cup\dots\cup S_q.\]
Согласно следствию~\ref{cor:Sq},
\[\left|\sum_{n\in S}s_n\cdot\theta_n\right|\le 2\cdot q\cdot(q+1)\cdot\pi.\eqlbl{sum-S}\]

Обозначим через $R$ множество индексов, не входящих в $s$-пары,
то есть $R=\{1,\dots,k\}\backslash S$.

Для целого $r$, запишем  $i\in Q_r$, 
если
\[\sum_{n=1}^i s_n=r.\]
Заметим, что $Q_r\ne\emptyset$ для не более чем $q$ значений $r$
и каждое множество $Q_r$ содержит не более $2$ индексов не из $s$-пар;
то есть $Q_r\cap R$ содержит не более двух индексов для каждого $r$.

Поскольку $|a_n|<\tfrac\pi2$, получаем
\[\left|\sum_{n\in R}s_n\cdot\theta_n\right|
\le
q\cdot\pi.
\]
Последнее неравенство вместе с \ref{sum-S} влечёт нужное утверждение.
\qeds



\section{Геометрический рост}\label{sec:geometric-growth}

\begin{thm}{Утверждение}\label{clm:alpha-psi}
Предположим, $\psi(t)>\eps$ для $t\in[t_{i},t_{i+1}]$ и $s_i=s_{i+1}$.
Тогда 
\[|\theta_{i+1}-\theta_i|>\pi\cdot\sin\eps.\] 

\end{thm}

\parit{Доказательство.}
Заметим, что дуга $\gamma|_{[t_{i},t_{i+1}]}$ является языком, пусть $\iota\:\DD^2\zz\to \Sigma$ его диск.
Поскольку $\psi(t)>\eps$, 
сферический образ $\nu\circ\iota(\DD^2)$ 
образа $\iota(\DD^2)$ 
лежит в полукруге радиуса $\tfrac\pi2-\eps$ в $\SS^2$.
Заметим, что
\[K(\iota(\DD^2))
=
\area(\nu\circ\iota(\DD^2))
<
\pi\cdot(1-\sin\eps).\]
Остаётся применить лемму о языке~\ref{lem:tongue}.
\qeds



\begin{thm}{Утверждение}\label{clm:geometric-grouth}
Предположим $\gamma$ лежит на тёмной стороне для $\bm{i}$.
Тогда для любой пары индексов $j>i$,
таких, что
\[\biggl|\sum_{n=i}^{j}s_n\biggr|> 5\]
выполняется
\[\phi_j>\tfrac32\cdot\phi_i.\]

\end{thm}


\parit{Доказательство.}
Согласно утверждению~\ref{clm:7.3}, можно предположить, что 
\[\sum_{n=i}^{j}s_n= 6.\]

Пусть $j'$ --- последний индекс такой, что
\[\biggl|\sum_{n=i}^{j'} s_n\biggr|=5.\]

В частности $\gamma|_{[t_i,t_{j'}]}$ обходит вокруг оси $x$ хотябы дважды.
Значит для любого $b\ge t_j$ существует значение $a\in[t_i,t_{j'}]$,
такое, что $[a,b]$ удовлетворяет условиям в \ref{prop:phi-psi}.
В частности, $\psi(b)>\phi_i$ для любого $b>t_j$.
Применяя утверждение~\ref{clm:alpha-psi},
получаем, что $|\theta_j|>\tfrac\pi2\cdot \phi_i$ или
$|\theta_{j'}|\zz>\tfrac\pi2\cdot \phi_i$.
По утверждению~\ref{clm:7.3}, $\phi_n$ неубывающая и $\phi_n\ge |\theta_n|$ для каждого $n$.
В обоих случаях
\[\phi_j>\tfrac\pi2\cdot \phi_i.\]
Отсюда результат.
\qeds



\begin{thm}{Предложение}\label{prop:graph}
Если $\gamma$ является $\bm{i}$-дрейфующей кратчайшей на тёмной стороне для $\bm{i}$, тогда
\[\tc_{\bm{j}}\gamma\le 100\cdot\pi.\]

\end{thm}

\parit{Доказательство.}
Можно предположить, что
$\gamma$ пересекает $\bm{j}$-горизонт $\omega_{\bm{j}}$ трансверсально.
Пусть $t_0<\dots<t_k$ --- моменты встреч $\gamma$ с $\omega_{\bm{j}}$ и
$s_0,\dots,s_k$ --- знаки.

Напомним, что $S_q$ обозначает подмножество индексов $\{1,\dots,k\}$ из $s$-пар с глубиной $q$.
По следствию~\ref{cor:Sq},
\[\left|\sum_{n\in S_q}s_n\cdot\theta_n\right|\le 4\cdot q\cdot \pi.\]
В частности,
\[\left|\sum_{n\in S_1\cup\dots\cup S_5}
s_n\cdot\theta_n\right|
\le 
40\cdot\pi.\]

Введём обозначение $R=\{1,\dots,k\}\backslash (S_1\cup\dots\cup S_5)$ ---
это множество включает индексы из $s$-пар с глубиной хотя бы $6$, 
а также индексы, не входящие в $s$-пары.

По утверждению~\ref{clm:alpha-phi-psi},
\[\left|\sum_{n\in R}
s_n\cdot\theta_n\right|
\le
\sum_{n\in R}
|\theta_n|\le \sum_{n\in R}\phi_n.\] 
Для оценки последней суммы воспользуемся результатами из раздела~\ref{sec:geometric-growth}.
Сначала подразобьём $R$ на 5 подмножеств $R_1,\dots,R_5$, 
полагая $n\in R_m$ если $m\equiv n\pmod 5$.

Для $n\in R_m$ обозначим через $n'$ следующий индекс в $R_m$.
Значение $n'$ определено для всех $n\in R_m$, кроме последнего.
Согласно утверждению~\ref{clm:geometric-grouth}, 
$\phi_{n'}>\tfrac32\cdot \phi_n$.
То есть последовательность $(\phi_n)_{n\in R_m}$ растёт быстрее чем геометрическая с коэффициентом  $\tfrac32$.
Поскольку $\phi_n$ не убывает по $n$,
получаем
\[\sum_{n\in R_m}\phi_n< 3\cdot\phi_k.\]
Отсюда
\[\sum_{n\in R}\phi_n< 15\cdot\phi_k\le\tfrac{15}2\cdot\pi.\]

По следствию~\ref{cor:liberman},
\begin{align*}
\tc_{\bm{j}}\gamma
&\le 
2\cdot\pi
+
2\cdot [\alpha_0-\alpha_1+\dots+(-1)^k\cdot\alpha_k]< 
\\
&< 100\cdot \pi.
\end{align*}
\qeds



\section{Сборка доказательства}


Пусть $\gamma\:[0,\ell]\to \Sigma$ --- кратчайшая на выпуклой поверхности $\Sigma\subset \RR^3$.

По предложениям  \ref{prop:semicontinuity} и \ref{prop:convegence} можно предположить, что $\Sigma$  гладкая, сильно выпуклая и замкнутая,
а кратчайшая $\gamma$ имеет конечную длину.

Согласно предложению~\ref{prop:almost-const}, 
можно выбрать $(x,y,z)$-координатную систему 
и перейти к $\bm{i}$-дрейфующей дуге $\gamma'$
кратчайшей $\gamma$ такой, что
\[\tc\gamma'>\tfrac{1}{100^{100}}\cdot\tc\gamma.
\eqlbl{eq:1/100}
\]
Мы будем использовать обозначения раздела~\ref{sec:drifting} для $\gamma'$.

Подвернув $(y,z)$-координатную плоскость, можно добиться того, что
\[\tc\gamma'\le 10\cdot\tc_{\bm{j}}\gamma'\]
и что $\gamma'$ пересекает горизонт $\omega_{\bm{j}}$ трансверсально.

Согласно следствию \ref{cor:around-twice},
можно подразбить $\gamma'$ на три дуги: 
\begin{itemize}
\item \emph{Левую} $\gamma_-'$ полностью лежащую на светлой стороне для $\bm{i}$.
\item \emph{Среднюю} $\gamma_0'$ обходящую ось $x$ не больше чем $4$ раза.
\item \emph{Правую} $\gamma_+'$ полностью лежащую на тёмной стороне для $\bm{i}$,
\end{itemize}

Действительно, выберем дугу $\gamma'|_{[a,b]}$  
справа от $(y,z)$-плоскости через
$\tfrac12\cdot(\gamma'(0)+\gamma'(\ell))$,
которая делает два оборота вокруг оси $x$ 
и предположим, что значение $b$ наименьшее возможное.  
Заметим, что если $\gamma'(s)$ лежит на $(y,z)$-плоскости через $\tfrac12\cdot(\gamma'(0)+\gamma'(\ell))$
тогда $[s,b]\supset[a,b]$ 
и любая поддуга $[s,b]$ обходит вокруг оси $x$ не больше двух раз.

По следствию~\ref{cor:around-twice}, 
можно взять $\gamma'_+=\gamma'|_{[b,\ell]}$.
(В случае если интервал $[a,b]$ не определён, 
мы предполагаем, что дуги $\gamma_+'$ нет.)

Повторив построение для обратного направления оси $x$,
получим левую дугу $\gamma_-'$.
Оставшейся кусок обозначим $\gamma_0'$.
Заметим, что дуга $\gamma_0'$ разделена $(y,z)$-плоскостью через $\tfrac12\cdot(\gamma'(0)+\gamma'(\ell))$ на две дуги, 
каждая из которых обходит ось $x$ не более двух раз. 
Таким образом число обходов $\gamma_0'$ не превышает $4$.

Оценим вариацию поворота для каждой дуги $\gamma'_-$, $\gamma'_0$ и $\gamma'_+$ отдельно.

По предложению~\ref{prop:graph}, получаем 
\[\tc_{\bm{j}}\gamma'_+\le 100\cdot\pi.\eqlbl{+}\]
Аналогично  
\[\tc_{\bm{j}}\gamma'_-\le 100\cdot\pi.\eqlbl{-}\]

По следствию~\ref{cor:gamma-0},
\[\tc_{\bm{j}}\gamma'_0\le 100\cdot\pi.\eqlbl{0}\]
Оценки \ref{+}, \ref{-} и \ref{0} вместе  дают
\[\tc_{\bm{j}}\gamma'\le 300\cdot\pi.\]
Остаётся применить \ref{eq:1/100}.
\qeds

\section{Заключительные замечания}

Заметим, что основная теорема становится не верной, если в её формулировке заменить \emph{кратчайшую} на \emph{геодезическую}, даже если ограничить длину геодезической внутренним диаметром поверхности.

В дополнение к следствию~\ref{cor:length-diameter},
мы знаем пару забавных свойств кратчайших на выпуклых поверхностях, которые отличают их от геодезических. 
Хотя они нам не помогли в доказательстве, мы хотели бы их упомянуть. 
Первое из этих свойств, по сути обнаружил Анатолий Милка в \cite{milka-bending}. 
Мы дадим формулировку  только для гладких поверхностей, поскольку она проще.

\begin{thm}{Лемма о выпрямлении}
Пусть $\gamma\:[0,\ell]\to\Sigma$ --- кратчайшая, параметризованная длиной,
от точки $p$ до $q$ на выпуклой гладкой гиперповерхности $\Sigma\subset\RR^m$ и $p_t=\gamma(t)-\dot\gamma(t)\cdot t$.
Тогда, для любого $t\in [0,\ell]$, точка $q$ лежит на тёмной стороне от $p_t$.
\end{thm}

\begin{thm}{Лемма о выпуклой горбушке}
Пусть $\Sigma\subset\RR^m$ --- замкнутая выпуклая гиперповерхность и
$\Pi$ --- гиперплоскость, разрезающая $\Sigma$ на две части $\Sigma_1$ и $\Sigma_2$.
Предположим, что отражение $\Sigma_1$ в плоскости $\Pi$ заключено внутри  $\Sigma$.
Тогда $\Sigma_1$ образует выпуклое множество в $\Sigma$,
то есть любая кратчайшая в $\Sigma$ с концами в $\Sigma_1$ полностью лежит в $\Sigma_1$.
\end{thm}

Доказательства обоих утверждений можно найти в \cite{petrunin-orthodox}.

\medskip

Стыдно признаться, но мы не смогли обобщить основную теорему на старшие размерности,
то есть мы не знаем ответ на следующий вопрос.

\begin{thm}{Открытый вопрос}
Верно ли, что для любого положительного целого $m$,
существует вещественная константа $C_m$,
такая, что вариация поворота произвольной кратчайшей на выпуклой замкнутой гиперповерхности в $\RR^m$ не превосходит $C_m$?
\end{thm}






\begin{thebibliography}{99}
\bibitem{AH-PSV}
Agarwal, P. K.; Har-Peled, S.; Sharir, M.; Varadarajan, K. R.,
``Approximating shortest paths on a convex polytope in three dimensions''.
\textit{J. ACM}
44.4 (1997),
pp. 567--584.
\bibitem{pach}
Pach, J.,
``Folding and turning along geodesics in a convex surface'',
\textit{Geombinatorics}
7.2 (1997)
pp. 61--65.
\bibitem{BKZ}
B{\'a}r{\'a}ny, I.; Kuperberg, K.; Zamfirescu, T.,
``Total curvature and spiralling shortest paths''.
\textit{Discrete Comput. Geom.}
30.2 (2003),
pp. 167--176.
\bibitem{liberman}
Либерман И. М.,
«Геодезические
 линии
 на
 выпуклых
 поверхностях»,
\textit{ДАН СССР}
32.5 (1941),
с. 310---313. 

\bibitem{usov}
Усов, В. В. 
«О длине сферического изображения геодезической на выпуклой поверхности». \textit{Сибирский математический журнал} 
17.1 (1976), 
с. 233---236.

\bibitem{berg}
Berg, I. D. 
"An estimate on the total curvature of a geodesic in Euclidean 3-space-with-boundary." 
\textit{Geometriae Dedicata} 
13.1 (1982),
pp. 1--6.

\bibitem{pogorelov}
Погорелов, А. В., 
\textit{Внешняя геометрия выпуклых поверхностей.} 
1969.

\bibitem{zalgaller}
Залгаллер, В. А. 
«Вопрос о сферическом изображении кратчайшей».
\textit{Укр. геометрический сб.}
10 (1971) 
с. 12---18.

\bibitem{milka}
Милка, А. Д. 
«Кратчайшая с неспрямляемым сферическим изображением». 
\textit{Укр. геометрический сб.} 16 (1974)
с. 35---52.

\bibitem{usov-conj-pog} 
Усов, В. В. 
«О пространственном повороте кривых на выпуклых поверхностях». 
\textit{Сибирский математический журнал}
17.6 (1976),
с. 1427---1430.

\bibitem{milka-liberman}
Милка, А. Д. 
«Аналог теоремы Либермана в римановом пространстве».
Украинский геометрический сборник,
24 (1981), 
с. 82---84,

\bibitem{petrunin}
Petrunin, A.
``Applications of quasigeodesics and gradient curves''.
\textit{Math. Sci. Res. Inst. Publ.}
30 (1997),
pp. 203--219

\bibitem{milka-bending}  Милка, А. Д. «Кратчайшие  линии  на  выпуклых  поверхностях».  Докл.  АН  СССР,   1979,   248, \textnumero1,  34---36  

\bibitem{petrunin-orthodox}
Petrunin, A.
``Exercises in Orthodox Geometry''.
\texttt{arXiv:0906.0290 [math.HO]}
\end{thebibliography}
\Addresses
\end{document}

\usepackage[
sorting=none,
bibencoding=auto,
backend=biber,
maxnames=10,
%babel=other,
autolang=other,
doi=false,
url=false,
style=numeric-comp,
isbn=false]{biblatex}
\usepackage{csquotes}
\AtEveryBibitem{\clearlist{language}}
\AtEveryBibitem{\clearlist{note}}
\renewbibmacro{in:}{}
%\renewcommand*{\bibfont}{\small}
\addbibresource{lib.bib}

\sloppy
\printbibliography[heading=bibintoc]
\fussy
