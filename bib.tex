\begin{thebibliography}{99}
\bibitem{AH-PSV}
Agarwal, P. K.; Har-Peled, S.; Sharir, M.; Varadarajan, K. R.,
``Approximating shortest paths on a convex polytope in three dimensions''.
\textit{J. ACM}
44.4 (1997),
pp. 567--584.
\bibitem{pach}
Pach, J.,
``Folding and turning along geodesics in a convex surface'',
\textit{Geombinatorics}
7.2 (1997)
pp. 61--65.
\bibitem{BKZ}
B{\'a}r{\'a}ny, I.; Kuperberg, K.; Zamfirescu, T.,
``Total curvature and spiralling shortest paths''.
\textit{Discrete Comput. Geom.}
30.2 (2003),
pp. 167--176.
\bibitem{liberman}
Либерман И. М.,
«Геодезические
 линии
 на
 выпуклых
 поверхностях»,
\textit{ДАН СССР}
32.5 (1941),
с. 310---313. 

\bibitem{usov}
Усов, В. В. 
«О длине сферического изображения геодезической на выпуклой поверхности». \textit{Сибирский математический журнал} 
17.1 (1976), 
с. 233---236.

\bibitem{berg}
Berg, I. D. 
"An estimate on the total curvature of a geodesic in Euclidean 3-space-with-boundary." 
\textit{Geometriae Dedicata} 
13.1 (1982),
pp. 1--6.

\bibitem{pogorelov}
Погорелов, А. В., 
\textit{Внешняя геометрия выпуклых поверхностей.} 
1969.

\bibitem{zalgaller}
Залгаллер, В. А. 
«Вопрос о сферическом изображении кратчайшей».
\textit{Укр. геометрический сб.}
10 (1971) 
с. 12---18.

\bibitem{milka}
Милка, А. Д. 
«Кратчайшая с неспрямляемым сферическим изображением». 
\textit{Укр. геометрический сб.} 16 (1974)
с. 35---52.

\bibitem{usov-conj-pog} 
Усов, В. В. 
«О пространственном повороте кривых на выпуклых поверхностях». 
\textit{Сибирский математический журнал}
17.6 (1976),
с. 1427---1430.

\bibitem{milka-liberman}
Милка, А. Д. 
«Аналог теоремы Либермана в римановом пространстве».
Украинский геометрический сборник,
24 (1981), 
с. 82---84,

\bibitem{petrunin}
Petrunin, A.
``Applications of quasigeodesics and gradient curves''.
\textit{Math. Sci. Res. Inst. Publ.}
30 (1997),
pp. 203--219

\bibitem{milka-bending}  Милка, А. Д. «Кратчайшие  линии  на  выпуклых  поверхностях».  Докл.  АН  СССР,   1979,   248, \textnumero1,  34---36  

\bibitem{petrunin-orthodox}
Petrunin, A.
``Exercises in Orthodox Geometry''.
\texttt{arXiv:0906.0290 [math.HO]}
\end{thebibliography}